

\newcommand{\num}[1]{\marginpar{#1}}

{%
\chapter{Kooperation}\label{kapitel.-kooperation}}

{%
\section{1. Doppelter Ursprung der
Manufaktur\label{doppelter-ursprung-der-manufaktur}}

\num{356} Die auf Teilung der Arbeit beruhende
Kooperation schafft sich ihre klassische Gestalt in der Manufaktur. Als
charakteristische Form des kapitalistischen Produktionsprozesses
herrscht sie vor während der eigentlichen Manufakturperiode, die, rauh
angeschlagen, von Mitte des 16. Jahrhunderts bis zum letzten Drittel des
achtzehnten währt.

{%
\subsection{Die Manufaktur entspringt auf doppelte
Weise.}\label{die-manufaktur-entspringt-auf-doppelte-weise.}}

Entweder werden Arbeiter von verschiedenartigen, selbständigen
Handwerken, durch deren Hände ein Produkt bis zu seiner letzten Reife
laufen muß, in eine Werkstatt unter dem Kommando desselben Kapitalisten
vereinigt. Z.B. eine Kutsche war das Gesamtprodukt der Arbeiten einer
großen Anzahl unabhängiger Handwerker, wie Stellmacher, Sattler,
Schneider, Schlosser, Gürtler, Drechsler, Posamentierer, Glaser, Maler,
Lackierer, Vergolder usw. Die Kutschenmanufaktur vereinigt alle diese
verschiednen Handwerker in ein Arbeitshaus, wo sie einander gleichzeitig
in die Hand arbeiten. Man kann eine Kutsche zwar nicht vergolden, bevor
sie gemacht ist. Werden aber viele Kutschen gleichzeitig gemacht, so
kann ein Teil beständig vergoldet werden, während ein andrer Teil eine
frühre Phase des Produktionsprozesses durchläuft. Soweit stehn wir noch
auf dem Boden der einfachen Kooperation, die ihr Material an Menschen
und Dingen vorfindet. Indes tritt sehr bald eine wesentliche Veränderung
ein. Der Schneider, Schlosser, Gürtler usw., der nur im Kutschenmachen
beschäftigt ist, verliert nach und nach mit der Gewohnheit auch die
Fähigkeit, sein altes Handwerk in seiner ganzen Ausdehnung zu betreiben.
Andrerseits erhält sein vereinseitigtes Tun jetzt die zweckmäßigste Form
für die verengte Wirkungssphäre. Ursprünglich erschien die
Kutschenmanufaktur als eine Kombination selbständiger Handwerke. Sie
wird allmählich Teilung der \num{357}
Kutschenproduktion in ihre verschiednen Sonderoperationen, wovon jede
einzelne zur ausschließlichen Funktion eines Arbeiters kristallisiert
und deren Gesamtheit vom Verein dieser Teilarbeiter verrichtet wird.
Ebenso entstand die Tuchmanufaktur und eine ganze Reihe andrer
Manufakturen aus der Kombination verschiedner Handwerke unter Kommando
desselben Kapitals.(26)

Die Manufaktur entspringt aber auch auf entgegengesetztem Wege. Es
werden viele Handwerker, die dasselbe oder Gleichartiges tun, z.B.
Papier oder Typen oder Nadeln machen, von demselben Kapital gleichzeitig
in derselben Werkstatt beschäftigt. Es ist dies Kooperation in der
einfachsten Form. Jeder dieser Handwerker (vielleicht mit einem oder
zwei Gesellen) macht die ganze Ware und vollbringt also die
verschiednen, zu ihrer Herstellung erheischten Operationen der Reihe
nach. Er arbeitet in seiner alten handwerksmäßigen Weise fort. Indes
veranlassen bald äußere Umstände, die Konzentration der Arbeiter in
demselben Raum und die Gleichzeitigkeit ihrer Arbeiten anders zu
vernutzen. Es soll z.B. ein größeres Quantum fertiger Ware in einer
bestimmten Zeitfrist geliefert werden. Die Arbeit wird daher verteilt.
Statt die verschiednen Operationen von demselben Handwerker in einer
zeitlichen Reihenfolge verrichten zu lassen, werden sie voneinander
losgelöst, isoliert, räumlich nebeneinander gestellt, jede derselben
einem andren Handwerker zugewiesen und alle zusammen von den
Kooperierenden gleichzeitig ausgeführt. Diese zufällige Verteilung
wiederholt sich, zeigt ihre eigentümlichen Vorteile und verknöchert nach
und nach zur systematischen Teilung der Arbeit. Aus dem indivi-
\num{358} duellen Produkt eines selbständigen
Handwerkers, der vielerlei tut, verwandelt sich die Ware in das
gesellschaftliche Produkt eines Vereins von Handwerkern, von denen jeder
fortwährend nur eine und dieselbe Teiloperation verrichtet. Dieselben
Operationen, die ineinander flossen als sukzessive Verrichtungen des
deutschen zünftigen Papiermachers, verselbständigten sich in der
holländischen Papiermanufaktur zu nebeneinander laufenden
Teiloperationen vieler kooperierenden Arbeiter. Der zünftige Nadler von
Nürnberg bildet das Grundelement der englischen Nadelmanufaktur. Während
aber jener eine Nadler eine Reihe von vielleicht 20 Operationen
nacheinander durchlief, verrichteten hier bald 20 Nadler nebeneinander,
jeder nur eine der 20 Operationen, die infolge von Erfahrungen noch viel
weiter gespaltet, isoliert und zu ausschließlichen Funktionen einzelner
Arbeiter verselbständigt wurden.

Die Ursprungsweise der Manufaktur, ihre Herausbildung aus dem Handwerk
ist also zwieschlächtig. Einerseits geht sie von der Kombination
verschiedenartiger, selbständiger Handwerke aus, die bis zu dem Punkt
verunselbständigt und vereinseitigt werden, wo sie nur noch einander
ergänzende Teiloperationen im Produktionsprozeß einer und derselben Ware
bilden. Andrerseits geht sie von der Kooperation gleichartiger
Handwerker aus, zersetzt dasselbe individuelle Handwerk in seine
verschiednen besondren Operationen und isoliert und verselbständigt
diese bis zu dem Punkt, wo jede derselben zur ausschließlichen Funktion
eines besondren Arbeiters wird. Einerseits führt daher die Manufaktur
Teilung der Arbeit in einen Produktionsprozeß ein oder entwickelt sie
weiter, andrerseits kombiniert sie früher geschiedne Handwerke. Welches
aber immer ihr besondrer Ausgangspunkt, ihre Schlußgestalt ist dieselbe
- ein Produktionsmechanismus, dessen Organe Menschen sind.

Zum richtigen Verständnis der Teilung der Arbeit in der Manufaktur ist
es wesentlich, folgende Punkte festzuhalten: Zunächst fällt die Analyse
des Produktionsprozesses in seine besondren Phasen hier ganz und gar
zusammen mit der Zersetzung einer handwerksmäßigen Tätigkeit in ihre
verschiednen Teiloperationen. Zusammengesetzt oder einfach, die
Verrichtung bleibt handwerksmäßig und daher abhängig von Kraft,
Geschick, Schnelle, Sicherheit des Einzelarbeiters in Handhabung seines
Instruments. Das Handwerk bleibt die Basis. Diese enge technische Basis
schließt wirklich wissenschaftliche Analyse des Produktionsprozesses
aus, da jeder Teilprozeß, den das Produkt durchmacht, als
handwerksmäßige Teilarbeit ausführbar sein muß. Eben weil das
handwerksmäßige Geschick so die Grundlage des Produktionsprozesses
bleibt, wird jeder Arbeiter ausschließ- \num{359}
lich einer Teilfunktion angeeignet und seine Arbeitskraft in das
lebenslängliche Organ dieser Teilfunktion verwandelt. Endlich ist diese
Teilung der Arbeit eine besondre Art der Kooperation, und manche ihrer
Vorteile entspringen aus dem allgemeinen Wesen, nicht aus dieser
besondren Form der Kooperation.

{%
\section{2. Der Teilarbeiter und sein
Werkzeug}\label{der-teilarbeiter-und-sein-werkzeug}}

Gehn wir nun näher auf das einzelne ein, so ist zunächst klar, daß ein
Arbeiter, der lebenslang eine und dieselbe einfache Operation
verrichtet, seinen ganzen Körper in ihr automatisch einseitiges Organ
verwandelt und daher weniger Zeit dazu verbraucht als der Handwerker,
der eine ganze Reihe von Operationen abwechselnd ausführt. Der
kombinierte Gesamtarbeiter, der den lebendigen Mechanismus der
Manufaktur bildet, besteht aber aus lauter solchen einseitigen
Teilarbeitern. Im Vergleich zum selbständigen Handwerk wird daher mehr
in weniger Zeit produziert oder die Produktivkraft der Arbeiter
gesteigert.(27) Auch vervollkommnet sich die Methode der Teilarbeit,
nachdem sie zur ausschließlichen Funktion einer Person verselbständigt
ist. Die stete Wiederholung desselben beschränkten Tuns und die
Konzentration der Aufmerksamkeit auf dieses Beschränkte lehren
erfahrungsmäßig den bezweckten Nutzeffekt mit geringstem Kraftaufwand
erreichen. Da aber immer verschiedne Arbeitergenerationen gleichzeitig
zusammenleben und in denselben Manufakturen zusammenwirken, befestigen,
häufen und übertragen sich bald die so gewonnenen technischen
Kunstgriffe.(28)

Die Manufaktur produziert in der Tat die Virtuosität des
Detailarbeiters, indem sie die naturwüchsige Sonderung der Gewerbe, die
sie in der Gesellschaft vorfand, im Innern der Werkstatt reproduziert
und systematisch zum Extrem treibt. Andrerseits entspricht ihre
Verwandlung der Teilarbeit in den Lebensberuf eines Menschen dem Trieb
früherer Gesellschaften, die Gewerbe erblich zu machen, sie in Kasten zu
versteinern oder in Zünfte zu verknöchern, falls bestimmte historische
\num{360} Bedingungen dem Kastenwesen widersprechende
Variabilität des Individuums erzeugen. Kasten und Zünfte entspringen aus
demselben Naturgesetz, welches die Sonderung von Pflanzen und Tieren in
Arten und Unterarten regelt, nur daß auf einem gewissen Entwicklungsgrad
die Erblichkeit der Kasten oder die Ausschließlichkeit der Zünfte als
gesellschaftliches Gesetz dekretiert wird.(29)

``Die Musline von Dakka sind an Feinheit, die Kattune und andre Zeuge
von Koromandel an Pracht und Dauerhaftigkeit der Farben niemals
übertroffen worden. Und dennoch werden sie produziert ohne Kapital,
Maschinerie, Teilung der Arbeit oder irgendeins der andren Mittel, die
der Fabrikation in Europa so viele Vorteile bieten. Der Weber ist ein
vereinzeltes Individuum, der das Gewebe auf Bestellung eines Kunden
verfertigt und mit einem Webstuhl von der einfachsten Konstruktion,
manchmal nur bestehend aus hölzernen, roh zusammengefügten Stangen. Er
besitzt nicht einmal einen Apparat zum Aufziehn der Kette, der Webstuhl
muß daher in seiner ganzen Länge ausgestreckt bleiben und wird so
unförmlich und weit, daß er keinen Raum findet in der Hütte des
Produzenten, der seine Arbeit daher in freier Luft verrichten muß, wo
sie durch jede Wetterändrung unterbrochen wird.''(30)

Es ist nur das von Generation auf Generation gehäufte und von Vater auf
Sohn vererbte Sondergeschick, das dem Hindu wie der Spinne diese
Virtuosität verleiht. Und dennoch verrichtet ein solcher indischer Weber
sehr komplizierte Arbeit, verglichen mit der Mehrzahl der
Manufakturarbeiter.

Ein Handwerker, der die verschiednen Teilprozesse in der Produktion
eines Machwerks nacheinander ausführt, muß bald den Platz, bald die
Instrumente wechseln. Der Übergang von einer Operation zur andren
\num{361} unterbricht den Fluß seiner Arbeit und
bildet gewissermaßen Poren in seinem Arbeitstag. Diese Poren verdichten
sich, sobald er den ganzen Tag eine und dieselbe Operation
kontinuierlich verrichtet, oder sie verschwinden in dem Maße, wie der
Wechsel seiner Operation abnimmt. Die gesteigerte Produktivität ist hier
entweder der zunehmenden Ausgabe von Arbeitskraft in einem gegebnen
Zeitraum geschuldet, also wachsender Intensität der Arbeit oder einer
Abnahme des unproduktiven Verzehrs von Arbeitskraft. Der Überschuß von
Kraftaufwand nämlich, den jeder Übergang aus der Ruhe in die Bewegung
erheischt, kompensiert sich bei längrer Fortdauer der einmal erreichten
Normalgeschwindigkeit. Andrerseits zerstört die Kontinuität
gleichförmiger Arbeit die Spann- und Schwungkraft der Lebensgeister, die
im Wechsel der Tätigkeit selbst ihre Erholung und ihren Reiz finden.

Die Produktivität der Arbeit hängt nicht nur von der Virtuosität des
Arbeiters ab, sondern auch von der Vollkommenheit seiner Werkzeuge.
Werkzeuge derselben Art, wie Schneider-, Bohr-, Stoß-, Schlaginstrumente
usw., werden in verschiednen Arbeitsprozessen gebraucht, und in
demselben Arbeitsprozeß dient dasselbe Instrument zu verschiednen
Verrichtungen. Sobald jedoch die verschiednen Operationen eines
Arbeitsprozesses voneinander losgelöst sind und jede Teiloperation in
der Hand des Teilarbeiters eine möglichst entsprechende und daher
ausschließliche Form gewinnt, werden Verändrungen der vorher zu
verschiednen Zwecken dienenden Werkzeuge notwendig. Die Richtung ihres
Formwechsels ergibt sich aus der Erfahrung der besondren
Schwierigkeiten, welche die unveränderte Form in den Weg legt. Die
Differenzierung der Arbeitsinstrumente, wodurch Instrumente derselben
Art besondre feste Formen für jede besondre Nutzanwendung erhalten, und
ihre Spezialisierung, wodurch jedes solches Sonderinstrument nur in der
Hand spezifischer Teilarbeiter in seinem ganzen Umfang wirkt,
charakterisieren die Manufaktur. Zu Birmingham allein produziert man
etwa 500 Varietäten von Hämmern, wovon jeder nicht nur für einen
besondren Produktionsprozeß, sondern eine Anzahl Varietäten oft nur für
verschiedne Operationen in demselben Prozeß dient. Die Manufakturperiode
vereinfacht, verbessert und vermannigfacht die Arbeitswerkzeuge durch
deren Anpassung an die ausschließlichen Sonderfunktionen der
Teilarbeiter.(31) Sie schafft damit zu- \num{362}
gleich eine der materiellen Bedingungen der Maschinerie, die aus einer
Kombination einfacher Instrumente besteht.

Der Detailarbeiter und sein Instrument bilden die einfachen Elemente der
Manufaktur. Wenden wir uns jetzt zu ihrer Gesamtgestalt.

{%
\section{3. Die beiden Grundformen der Manufaktur
-}\label{die-beiden-grundformen-der-manufaktur--}}

heterogene Manufaktur und organische Manufaktur

Die Gliederung der Manufaktur besitzt zwei Grundformen, die trotz
gelegentlicher Verschlingung zwei wesentlich verschiedne Arten bilden
und namentlich auch bei der spätren Verwandlung der Manufaktur in die
maschinenartig betriebne, große Industrie eine ganz verschiedne Rolle
spielen. Dieser Doppelcharakter entspringt aus der Natur des Machwerks
selbst. Es wird entweder gebildet durch bloß mechanische Zusammensetzung
selbständiger Teilprodukte oder verdankt seine fertige Gestalt einer
Reihenfolge zusammenhängender Prozesse und Manipulationen.

Eine Lokomotive z.B. besteht aus mehr als 5.000 selbständigen Teilen.
Sie kann jedoch nicht als Beispiel der ersten Art der eigentlichen
Manufaktur gelten, weil sie ein Gebilde der großen Industrie ist. Wohl
aber die Uhr, an welcher auch William Petty die manufakturmäßige Teilung
der Arbeit veranschaulicht. Aus dem individuellen Werk eines Nürnberger
Handwerkers verwandelte sich die Uhr in das gesellschaftliche Produkt
einer Unzahl von Teilarbeitern, wie Rohwerkmacher, Uhrfedermacher,
Zifferblattmacher, Spiralfedermacher, Steinloch- und Rubinhebelmacher,
Zeigermacher, Gehäusemacher, Schraubenmacher, Vergolder, mit vielen
Unterabteilungen, wie z.B. Räderfabrikant (Messing- und Stahlräder
wieder geschieden), Triebmacher, Zeigerwerkmacher, acheveur de pignon
(befestigt die Räder auf den Trieben, poliert die facettes usw.),
Zapfenmacher, planteur de finissage (setzt verschiedne Räder und Triebe
in das Werk), finisseur de barillet (läßt Zähne einschneiden, macht die
Löcher zur richtigen Weite, härtet Stellung und Gesperr), Hemmungmacher,
bei der Zylinderhemmung wieder Zylindermacher, Steigradmacher, Unruhe-
\num{363} macher, Requettemacher (das Rückwerk, woran
die Uhr reguliert wird), planteur d'échappement (eigentliche
Hemmungmacher); dann der repasseur de barillet (macht Federhaus und
Stellung ganz fertig), Stahlpolierer, Räderpolierer, Schraubenpolierer,
Zahlenmaler, Blattmacher (schmilzt das Email auf das Kupfer), fabricant
de pendants (macht bloß die Bügel des Gehäuses), finisseur de charnière
(steckt den Messingstift in die Mitte des Gehäuses etc.), faiseur de
secret (macht die Federn im Gehäuse, die den Deckel aufspringen machen),
graveur, ciseleur, polisseur de boîte usw., usw., endlich der repasseur,
der die ganze Uhr zusammensetzt und sie gehend abliefert. Nur wenige
Teile der Uhr laufen durch verschiedne Hände, und alle diese membra
disjecta sammeln sich erst in der Hand, die sie schließlich in ein
mechanisches Ganzes verbindet. Dies äußerliche Verhältnis des fertigen
Produkts zu seinen verschiedenartigen Elementen läßt hier, wie bei
ähnlichem Machwerk, die Kombination der Teilarbeiter in derselben
Werkstatt zufällig. Die Teilarbeiten können selbst wieder als
voneinander unabhängige Handwerke betrieben werden, wie im Kanton Waadt
und Neuchâtel, während in Genf z.B. große Uhrenmanufakturen bestehn,
d.h. unmittelbare Kooperation der Teilarbeiter unter dem Kommando eines
Kapitals stattfindet. Auch im letztren Fall werden Zifferblatt, Feder
und Gehäuse selten in der Manufaktur selbst verfertigt. Der kombinierte
manufakturmäßige Betrieb ist hier nur unter ausnahmsweisen Verhältnissen
profitlich, weil die Konkurrenz unter den Arbeitern, die zu Hause
arbeiten wollen, am größten ist, die Zersplittrung der Produktion in
eine Masse heterogener Prozesse wenig Verwendung gemeinschaftlicher
Arbeitsmittel erlaubt und der Kapitalist bei der zerstreuten Fabrikation
die Auslage für Arbeitsgebäude usw. erspart.(32) Indes ist auch die
Stellung dieser \num{364} Detailarbeiter, die zu
Hause, aber für einen Kapitalisten (Fabrikant, établisseur) arbeiten,
ganz und gar verschieden von der des selbständigen Handwerkers, welcher
für seine eignen Kunden arbeitet.(33)

Die zweite Art der Manufaktur, ihre vollendete Form, produziert
Machwerke, die zusammenhängende Entwicklungsphasen, eine Reihenfolge von
Stufenprozessen durchlaufen, wie z.B. der Draht in der
Nähnadelmanufaktur die Hände von 72 und selbst 92 spezifischen
Teilarbeitern durchläuft.

Soweit solche Manufaktur ursprünglich zerstreute Handwerke kombiniert,
vermindert sie die räumliche Trennung zwischen den besondren
Produktionsphasen des Machwerks. Die Zeit seines Übergangs aus einem
Stadium in das andre wird verkürzt, ebenso die Arbeit, welche diese
Übergänge vermittelt.(34) Im Vergleich zum Handwerk wird so
Produktivkraft gewonnen, und zwar entspringt dieser Gewinn aus dem
allgemeinen kooperativen Charakter der Manufaktur. Andrerseits bedingt
ihr eigentümliches Prinzip der Teilung der Arbeit eine Isolierung der
verschiednen Produktionsphasen, die als ebenso viele handwerksmäßige
Teilarbeiten gegeneinander verselbständigt sind. Die Herstellung und
Erhaltung des Zusammenhangs zwischen den isolierten Funktionen ernötigt
beständigen Transport des Machwerks aus einer Hand in die andre und aus
einem Prozeß in den andren. Vom Standpunkt der großen Industrie tritt
dies als eine charakteristische, kostspielige und dem Prinzip der
Manufaktur immanente Beschränktheit hervor.(35)

Betrachtet man ein bestimmtes Quantum Rohmaterial, z.B. von Lumpen in
der Papiermanufaktur oder von Draht in der Nadelmanufaktur, so
durchläuft es in den Händen der verschiednen Teilarbeiter eine zeitliche
Stufenfolge von Produktionsphasen bis zu seiner Schlußgestalt.
Betrachtet \num{365} man dagegen die Werkstatt als
einen Gesamtmechanismus, so befindet sich das Rohmaterial gleichzeitig
in allen seinen Produktionsphasen auf einmal. Mit einem Teil seiner
vielen instrumentbewaffneten Hände zieht der aus den Detailarbeiten
kombinierte Gesamtarbeiter den Draht, während er gleichzeitig mit andren
Händen und Werkzeugen ihn streckt, mit andren schneidet, spitzt etc. Aus
einem zeitlichen Nacheinander sind die verschiednen Stufenprozesse in
ein räumliches Nebeneinander verwandelt. Daher Lieferung von mehr
fertiger Ware in demselben Zeitraum.(36) Jene Gleichzeitigkeit
entspringt zwar aus der allgemeinen kooperativen Form des
Gesamtprozesses, aber die Manufaktur findet nicht nur die Bedingungen
der Kooperation vor, sondern schafft sie teilweise erst durch die
Zerlegung der handwerksmäßigen Tätigkeit. Andrerseits erreicht sie diese
gesellschaftliche Organisation des Arbeitsprozesses nur durch
Festschmieden desselben Arbeiters an dasselbe Detail.

Da das Teilprodukt jedes Teilarbeiters zugleich nur eine besondre
Entwicklungsstufe desselben Machwerks ist, liefert ein Arbeiter dem
andren oder eine Arbeitergruppe der andern ihr Rohmaterial. Das
Arbeitsresultat des einen bildet den Ausgangspunkt für die Arbeit des
andren. Der eine Arbeiter beschäftigt daher hier unmittelbar den andren.
Die notwendige Arbeitszeit zur Erreichung des bezweckten Nutzeffekts in
jedem Teilprozeß wird erfahrungsmäßig festgestellt, und der
Gesamtmechanismus der Manufaktur beruht auf der Voraussetzung, daß in
gegebner Arbeitszeit ein gegebnes Resultat erzielt wird. Nur unter
dieser Voraussetzung können die verschiednen, einander ergänzenden
Arbeitsprozesse ununterbrochen, gleichzeitig und räumlich nebeneinander
fortgehn. Es ist klar, daß diese unmittelbare Abhängigkeit der Arbeiten
und daher der Arbeiter voneinander jeden einzelnen zwingt, nur die
notwendige Zeit zu seiner Funktion zu verwenden, und so eine ganz andre
Kontinuität, Gleichförmigkeit, Regelmäßigkeit, Ordnung (37) und
namentlich auch Intensität der Arbeit \num{366}
erzeugt wird als im unabhängigen Handwerk oder selbst der einfachen
Kooperation. Daß auf eine Ware nur die zu ihrer Herstellung
gesellschaftlich notwendige Arbeitszeit verwandt wird, erscheint bei der
Warenproduktion überhaupt als äußrer Zwang der Konkurrenz, weil,
oberflächlich ausgedrückt, jeder einzelne Produzent die Ware zu ihrem
Marktpreis verkaufen muß. Lieferung von gegebnem Produktenquantum in
gegebner Arbeitszeit wird dagegen in der Manufaktur technisches Gesetz
des Produktionsprozesses selbst.(38)

Verschiedne Operationen bedürfen jedoch ungleicher Zeitlängen und
liefern daher in gleichen Zeiträumen ungleiche Quanta von Teilprodukten.
Soll also derselbe Arbeiter tagaus, tagein stets nur dieselbe Operation
verrichten, so müssen für verschiedne Operationen verschiedne
Verhältniszahlen von Arbeitern verwandt werden, z.B. 4 Gießer und 2
Abbrecher auf einen Frottierer in einer Typenmanufaktur, wo der Gießer
stündlich 2.000 Typen gießt, der Abbrecher 4.000 abbricht und der
Frottierer 8.000 blank reibt. Hier kehrt das Prinzip der Kooperation in
seiner einfachsten Form zurück, gleichzeitige Beschäftigung vieler, die
Gleichartiges tun, aber jetzt als Ausdruck eines organischen
Verhältnisses. Die manufakturmäßige Teilung der Arbeit vereinfacht und
vermannigfacht also nicht nur die qualitativ unterschiednen Organe des
gesellschaftlichen Gesamtarbeiters, sondern schafft auch ein
mathematisch festes Verhältnis für den quantitativen Umfang dieser
Organe, d.h. für die relative Arbeiterzahl oder relative Größe der
Arbeitergruppen in jeder Sonderfunktion. Sie entwickelt mit der
qualitativen Gliederung die quantitative Regel und Proportionalität des
gesellschaftlichen Arbeitsprozesses.

Ist die passendste Verhältniszahl der verschiednen Gruppen von
Teilarbeitern erfahrungsmäßig festgesetzt für eine bestimmte
Stufenleiter der Produktion, so kann man diese Stufenleiter nur
ausdehnen, indem man ein Multipel jeder besondren Arbeitergruppe
verwendet.(39) Es kommt hinzu, daß dasselbe Individuum gewisse Arbeiten
ebensogut auf größerer als \num{367} kleinerer
Staffel ausführt, z.B. die Arbeit der Oberaufsicht, den Transport der
Teilprodukte aus einer Produktionsphase in die andre usw. Die
Verselbständigung dieser Funktionen oder ihre Zuweisung an besondre
Arbeiter wird also erst vorteilhaft mit Vergrößrung der beschäftigten
Arbeiterzahl, aber diese Vergrößrung muß sofort alle Gruppen
proportionell ergreifen.

Die einzelne Gruppe, eine Anzahl von Arbeitern, die dieselbe
Teilfunktion verrichten, besteht aus homogenen Elementen und bildet ein
besondres Organ des Gesamtmechanismus. In verschiednen Manufakturen
jedoch ist die Gruppe selbst ein gegliederter Arbeitskörper, während der
Gesamtmechanismus durch die Wiederholung oder Vervielfältigung dieser
produktiven Elementarorganismen gebildet wird. Nehmen wir z.B. die
Manufaktur von Glasflaschen. Sie zerfällt in drei wesentlich
unterschiedne Phasen. Erstens die vorbereitende Phase, wie Bereitung der
Glaskomposition, Mengung von Sand, Kalk usw. und Schmelzung dieser
Komposition zu einer flüssigen Glasmasse.(40) In der ersten Phase sind
verschiedne Teilarbeiter beschäftigt, ebenso in der Schlußphase, der
Entfernung der Flaschen aus den Trockenöfen, ihrer Sortierung,
Verpackung usw. Zwischen beiden Phasen steht in der Mitte die
eigentliche Glasmacherei oder Verarbeitung der flüssigen Glasmasse. An
demselben Munde eines Glasofens arbeitet eine Gruppe, die in England das
``hole'' (Loch) heißt und aus einem bottle maker oder finischer, einem
blower, einem gatherer, einem putter up oder whetter off und einem taker
in \textless{}Flaschenmacher oder Fertigmacher, einem Bläser, einem
Anfänger, einem Aufstapler oder Absprenger und einem
Abträger\textgreater{} zusammengesetzt ist. Diese fünf Teilarbeiter
bilden ebenso viele Sonderorgane eines einzigen Arbeitskörpers, der nur
als Einheit, also nur durch unmittelbare Kooperation der fünf wirken
kann. Fehlt ein Glied des fünfteiligen Körpers, so ist er paralysiert.
Derselbe Glasofen hat aber verschiedne Öffnungen, in England z.B. 4-6,
deren jede einen irdenen Schmelztiegel mit flüssigem Glas birgt und
wovon jede eine eigne Arbeitergruppe von derselben fünfgliedrigen Form
beschäftigt. Die Gliederung jeder einzelnen Gruppe beruht hier
unmittelbar auf der Teilung der Arbeit, während das Band zwischen den
verschiednen gleichartigen Gruppen einfache Kooperation ist, die eins
der Produktionsmittel, hier den Glasofen, durch gemeinsamen
\num{368} Konsum ökonomischer verbraucht. Ein solcher
Glasofen mit seinen 4-6 Gruppen bildet eine Glashütte, und eine
Glasmanufaktur umfaßt eine Mehrzahl solcher Hütten, zugleich mit den
Vorrichtungen und Arbeitern für die einleitenden und abschließenden
Produktionsphasen.

Endlich kann die Manufaktur, wie sie teilweis aus der Kombination
verschiedner Handwerke entspringt, sich zu einer Kombination
verschiedner Manufakturen entwickeln. Die größren englischen Glashütten
z.B. fabrizieren ihre irdenen Schmelztiegel selbst, weil von deren Güte
das Gelingen oder Mißlingen des Produkts wesentlich abhängt. Die
Manufaktur eines Produktionsmittels wird hier mit der Manufaktur des
Produkts verbunden. Umgekehrt kann die Manufaktur des Produkts verbunden
werden mit Manufakturen, worin es selbst wieder als Rohmaterial dient
oder mit deren Produkten es später zusammengesetzt wird. So findet man
z.B. die Manufaktur von Flintglas kombiniert mit der Glasschleiferei und
der Gelbgießerei, letztre für die metallische Einfassung mannigfacher
Glasartikel. Die verschiednen kombinierten Manufakturen bilden dann mehr
oder minder räumlich getrennte Departemente einer Gesamtmanufaktur,
zugleich voneinander unabhängige Produktionsprozesses, jeder mit eigner
Teilung der Arbeit. Trotz mancher Vorteile, welche die kombinierte
Manufaktur bietet, gewinnt sie, auf eigner Grundlage, keine wirklich
technische Einheit. Diese entsteht erst bei ihrer Verwandlung in den
maschinenmäßigen Betrieb.

Die Manufakturperiode, welche Verminderung der zur Warenproduktion
notwendigen Arbeitszeit bald als bewußtes Prinzip ausspricht (41),
entwickelt sporadisch auch den Gebrauch von Maschinen, namentlich für
gewisse einfache erste Prozesse, die massenhaft und mit großem
Kraftaufwand auszuführen sind. So wird z.B. bald in der Papiermanufaktur
das Zermalmen der Lumpen durch Papiermühlen und in der Metallurgie das
Zerstoßen der Erze durch sogenannte Pochmühlen verrichtet.(42) Die
elementarische Form aller Maschinerie hatte das römische Kaiserreich
überliefert in der Wassermühle.(43) Die Handwerksperiode vermachte die
großen \num{369} Erfindungen des Kompasses, des
Pulvers, der Buchdruckerei und der automatischen Uhr. Im großen und
ganzen jedoch spielt die Maschinerie jene Nebenrolle, die Adam Smith ihr
neben der Teilung der Arbeit anweist.(44) Sehr wichtig wurde die
sporadische Anwendung der Maschinerie im 17. Jahrhundert, weil sie den
großen Mathematikern jener Zeit praktische Anhaltspunkte und Reizmittel
zur Schöpfung der modernen Mechanik darbot.

Die spezifische Maschinerie der Manufakturperiode bleibt der aus vielen
Teilarbeitern kombinierte Gesamtarbeiter selbst. Die verschiednen
Operationen, die der Produzent einer Ware abwechselnd verrichtet und die
sich im Ganzen seines Arbeitsprozesses verschlingen, nehmen ihn
verschiedenartig in Anspruch. In der einen muß er mehr Kraft entwickeln,
in der andren mehr Gewandtheit, in der dritten mehr geistige
Aufmerksamkeit usw., und dasselbe Individuum besitzt diese Eigenschaften
nicht in gleichem Grad. Nach der Trennung, Verselbständigung und
Isolierung der verschiednen Operationen werden die Arbeiter ihren
vorwiegenden Eigenschaften gemäß geteilt, klassifiziert und gruppiert.
Bilden ihre Naturbesonderheiten die Grundlage, worauf sich die Teilung
der Arbeit pfropft, so entwickelt die Manufaktur, einmal eingeführt,
Arbeitskräfte, die von Natur nur zu einseitiger Sonderfunktion taugen.
Der Gesamtarbeiter besitzt jetzt alle produktiven Eigenschaften in
gleich hohem Grad der Virtuosität und verausgabt sie zugleich aufs
ökonomischste, indem er alle seine Organe, individualisiert in besondren
Arbeitern oder Arbeitergruppen, ausschließlich zu ihren spezifischen
Funktionen verwendet.(45) Die \num{370} Einseitigkeit
und selbst die Unvollkommenheit des Teilarbeiters werden zu seiner
Vollkommenheit als Glied des Gesamtarbeiters.(46) Die Gewohnheit einer
einseitigen Funktion verwandelt ihn in ihr naturgemäß sicher wirkendes
Organ, während der Zusammenhang des Gesamtmechanismus ihn zwingt, mit
der Regelmäßigkeit eines Maschinenteils zu wirken.(47)

Da die verschiednen Funktionen des Gesamtarbeiters einfacher oder
zusammengesetzter, niedriger oder höher, erheischen seine Organe, die
individuellen Arbeitskräfte, sehr verschiedne Grade der Ausbildung und
besitzen daher sehr verschiedne Werte. Die Manufaktur entwickelt also
eine Hierarchie der Arbeitskräfte, der eine Stufenleiter der
Arbeitslöhne entspricht. Wird einerseits der individuelle Arbeiter einer
einseitigen Funktion angeeignet und lebenslang annexiert, so werden
ebensosehr die verschiednen Arbeitsverrichtungen jener Hierarchie der
natürlichen und erworbnen Geschicklichkeiten angepaßt.(48) Jeder
Produktionsprozeß bedingt indes gewisse einfache Hantierungen, deren
jeder Mensch, wie er geht und steht, fähig ist. Auch sie werden jetzt
von ihrem flüssigen Zusammenhang mit den inhaltvollern Momenten der
Tätigkeit losgelöst und zu ausschließlichen Funktionen verknöchert.

\num{371} Die Manufaktur erzeugt daher in jedem
Handwerk, das sie ergreift, eine Klasse sogenannter ungeschickter
Arbeiter, die der Handwerksbetrieb streng ausschloß. Wenn sie die
durchaus vereinseitigte Spezialität auf Kosten des ganzen
Arbeitsvermögens zur Virtuosität entwickelt, beginnt sie auch schon den
Mangel aller Entwicklung zu einer Spezialität zu machen. Neben die
hierarchische Abstufung tritt die einfache Scheidung der Arbeiter in
geschickte und ungeschickte. Für letztre fallen die Erlernungskosten
ganz weg, für erstre sinken sie, im Vergleich zum Handwerker, infolge
vereinfachter Funktion. In beiden Fällen sinkt der Wert der
Arbeitskraft.(49) Ausnahme findet statt, soweit die Zersetzung des
Arbeitsprozesses neue zusammenfassende Funktionen erzeugt, die im
Handwerksbetrieb gar nicht oder nicht in demselben Umfang vorkamen. Die
relative Entwertung der Arbeitskraft, die aus dem Wegfall oder der
Verminderung der Erlernungskosten entspringt, schließt unmittelbar
höhere Verwertung des Kapitals ein, denn alles, was die zur Reproduktion
der Arbeitskraft notwendige Zeit verkürzt, verlängert die Domäne der
Mehrarbeit.

{%
\section{4. Teilung der Arbeit innerhalb der
Manufaktur}\label{teilung-der-arbeit-innerhalb-der-manufaktur}}

und Teilung der Arbeit innerhalb der Gesellschaft

Wir betrachteten erst den Ursprung der Manufaktur, dann ihre einfachen
Elemente, den Teilarbeiter und sein Werkzeug, endlich ihren
Gesamtmechanismus. Wir berühren jetzt kurz das Verhältnis zwischen der
manufakturmäßigen Teilung der Arbeit und der gesellschaftlichen Teilung
der Arbeit, welche die allgemeine Grundlage aller Warenproduktion
bildet.

Hält man nur die Arbeit selbst im Auge, so kann man die Trennung der
gesellschaftlichen Produktion in ihre großen Gattungen, wie Agrikultur,
Industrie usw., als Teilung der Arbeit im allgemeinen, die Sonderung
dieser Produktionsgattungen in Arten und Unterarten als Teilung der
Arbeit im besondren, und die Teilung der Arbeit innerhalb einer
Werkstatt als Teilung der Arbeit im einzelnen bezeichnen.(50)

\num{372} Die Teilung der Arbeit innerhalb der
Gesellschaft und die entsprechende Beschränkung der Individuen auf
besondre Berufssphären entwickelt sich, wie die Teilung der Arbeit
innerhalb der Manufaktur, von entgegengesetzten Ausgangspunkten.
Innerhalb einer Familie (50a), weiter entwickelt eines Stammes,
entspringt eine naturwüchsige Teilung der Arbeit aus den Geschlechts-
und Alterverschiedenheiten, also auf rein physiologischer Grundlage, die
mit der Ausdehnung des Gemeinwesens, der Zunahme der Bevölkerung und
namentlich dem Konflikt zwischen verschiednen Stämmen und der
Unterjochung eines Stamms durch den andren ihr Material ausweitet.
Andrerseits, wie ich früher bemerkt , entspringt der Produktenaustausch
an den Punkten, wo verschiedne Familien, Stämme, Gemeinwesen in Kontakt
kommen, denn nicht Privatpersonen sondern Familien, Stämme usw. treten
sich in den Anfängen der Kultur selbständig gegenüber. Verschiedne
Gemeinwesen finden verschiedne Produktionsmittel und verschiedne
Lebensmittel in ihrer Naturumgebung vor. Ihre Produktionsweise,
Lebensweise und Produkte sind daher verschieden. Es ist diese
naturwüchsige Verschiedenheit, die bei dem Kontakt der Gemeinwesen den
Austausch der wechselseitigen Produkte und daher die allmähliche
Verwandlung dieser Produkte in Waren hervorruft. Der Austausch schafft
nicht den Unterschied der Produktionssphären, sondern setzt die
unterschiednen in Beziehung und verwandelt sie so in mehr oder minder
voneinander abhängige Zweige einer gesellschaftlichen Gesamtproduktion.
Hier entsteht die gesellschaftliche Teilung der Arbeit
\num{373} durch den Austausch ursprünglich
verschiedner, aber voneinander unabhängiger Produktionssphären. Dort, wo
die physiologische Teilung der Arbeit den Ausgangspunkt bildet, lösen
sich die besondren Organe eines unmittelbar zusammengehörigen Ganzen
voneinander ab, zersetzen sich, zu welchem Zersetzungsprozeß der
Warenaustausch mit fremden Gemeinwesen den Hauptanstoß gibt, und
verselbständigen sich bis zu dem Punkt, wo der Zusammenhang der
verschiednen Arbeiten durch den Austausch der Produkte als Waren
vermittelt wird. Es ist in dem einen Fall Verunselbständigung der früher
Selbständigen, in dem andren Verselbständigung der früher
Unselbständigen.

Die Grundlage aller entwickelten und durch Warenaustausch vermittelten
Teilung der Arbeit ist die Scheidung von Stadt und Land.(51) Man kann
sagen, daß die ganze ökonomische Geschichte der Gesellschaft sich in der
Bewegung dieses Gegensatzes resümiert, auf den wir jedoch hier nicht
weiter eingehn.

Wie für die Teilung der Arbeit innerhalb der Manufaktur eine gewisse
Anzahl gleichzeitig angewandter Arbeiter die materielle Voraussetzung
bildet, so für die Teilung der Arbeit innerhalb der Gesellschaft die
Größe der Bevölkerung und ihre Dichtigkeit, die hier an die Stelle der
Agglomeration in derselben Werkstatt tritt.(52) Indes ist diese
Dichtigkeit etwas Relatives. Ein relativ spärlich bevölkertes Land mit
entwickelten Kommunikationsmitteln besitzt eine dichtere Bevölkerung als
ein mehr bevölkertes Land mit unentwickelten Kommunikationsmitteln, und
in dieser Art sind z.B. die nördlichen Staaten der amerikanischen Union
dichter bevölkert als Indien.(53)

\num{374} Da Warenproduktion und Warenzirkulation die
allgemeine Voraussetzung der kapitalistischen Produktionsweise,
erheischt manufakturmäßige Teilung der Arbeit eine schon bis zu gewissem
Entwicklungsgrad gereifte Teilung der Arbeit im Innern der Gesellschaft.
Umgekehrt entwickelt und vervielfältigt die manufakturmäßige Teilung der
Arbeit rückwirkend jene gesellschaftliche Teilung der Arbeit. Mit der
Differenzierung der Arbeitsinstrumente differenzieren sich mehr und mehr
die Gewerbe, welche diese Instrumente produzieren.(54) Ergreift der
manufakturmäßige Betrieb ein Gewerb, das bisher als Haupt- oder
Nebengewerb mit andren zusammenhing und von demselben Produzenten
ausgeführt wurde, so findet sofort Scheidung und gegenseitige
Verselbständigung statt. Ergreift er eine besondre Produktionsstufe
einer Ware, so verwandeln sich ihre verschiednen Produktionsstufen in
verschiedne unabhängige Gewerbe. Es ward bereits angedeutet, daß, wo das
Machwerk ein bloß mechanisch zusammengesetztes Ganze von Teilprodukten,
die Teilarbeiten sich selbst wieder zu eignen Handwerken
verselbständigen können. Um die Teilung der Arbeiter vollkommner
innerhalb einer Manufaktur auszuführen, wird derselbe Produktionszweig,
je nach der Verschiedenheit seiner Rohstoffe oder der verschiednen
Formen, die derselbe Rohstoff erhalten kann, in verschiedne, zum Teil
ganz neue Manufakturen gespaltet. So wurden bereits in der ersten Hälfte
des 18. Jahrhunderts in Frankreich allein über 100 verschiedenartige
Seidenzeuge gewebt, und in Avignon z.B. war es Gesetz, daß ``jeder
Lehrling sich immer nur einer Fabrikationsart widmen und nicht die
Verfertigung mehrerer Zeugarten zugleich lernen durfte''. Die
territoriale Teilung der Arbeit, welche besondre Produktionszweige an
besondre Distrikte eines Landes bannt, erhält neuen Anstoß durch den
manufakturmäßigen Betrieb, der alle Besonderheiten ausbeutet.(55)
Reiches Material zur Teilung der Arbeit innerhalb der Gesellschaft
liefert der \num{375} Manufakturperiode die
Erweiterung des Weltmarkts und das Kolonialsystem, die zum Umkreis ihrer
allgemeinen Existenzbedingungen gehören. Es ist hier nicht der Ort,
weiter nachzuweisen, wie sie neben der ökonomischen jede andre Sphäre
der Gesellschaft ergreift und überall die Grundlage zu jener Ausbildung
des Fachwesens, der Spezialitäten, und einer Parzellierung des Menschen
legt, die schon A. Ferguson, den Lehrer A. Smiths, in den Ausruf
ausbrechen ließ: ``Wir machen eine Nation von Heloten, und es gibt keine
Freien unter uns.''(56)

Trotz der zahlreichen Analogien jedoch und der Zusammenhänge zwischen
der Teilung der Arbeit im Innern der Gesellschaft und der Teilung
innerhalb einer Werkstatt sind beide nicht nur graduell, sondern
wesentlich unterschieden. Am schlagendsten scheint die Analogie
unstreitig, wo ein innres Band verschiedne Geschäftszweige verschlingt.
Der Viehzüchter z.B. produziert Häute, der Gerber verwandelt die Häute
in Leder, der Schuster das Leder in Stiefel. Jeder produziert hier ein
Stufenprodukt, und die letzte fertige Gestalt ist das kombinierte
Produkt ihrer Sonderarbeiten. Es kommen hinzu die mannigfachen
Arbeitszweige, die dem Viehzüchter, Gerber, Schuster Produktionsmittel
liefern. Man kann sich nun mit A. Smith einbilden, diese
gesellschaftliche Teilung der Arbeit unterscheide sich von der
manufakturmäßigen nur subjektiv, nämlich für den Beobachter, der hier
die mannigfachen Teilarbeiten auf einen Blick räumlich zusammensieht,
während dort ihre Zerstreuung über große Flächen und die große Zahl der
in jedem Sonderzweig Beschäftigten den Zusammenhang verdunklen.(57) Was
aber stellt den Zusammenhang her \num{376} zwischen
den unabhängigen Arbeiten von Viehzüchter, Gerber, Schuster? Das Dasein
ihrer respektiven Produkte als Waren. Was charakterisiert dagegen die
manufakturmäßige Teilung der Arbeit? Daß der Teilarbeiter keine Ware
produziert.(58) Erst das gemeinsame Produkt der Teilarbeiter verwandelt
sich in Ware.(58a) Die Teilung der Arbeit im Innern der Gesellschaft ist
vermittelt durch den Kauf und Verkauf der Produkte verschiedner
Arbeitszweige, der Zusammenhang der Teilarbeiten in der Manufaktur durch
den Verkauf verschiedner Arbeitskräfte an denselben Kapitalisten, der
sie als kombinierte Arbeitskraft verwendet. Die manufakturmäßige Teilung
der Arbeit unterstellt Konzentration der Produktionsmittel in der Hand
eines Kapitalisten, die gesellschaftliche Teilung der Arbeit
Zersplitterung der Produktionsmittel unter viele voneinander unabhängige
Warenproduzenten. Statt daß in der Manufaktur das eherne Gesetz der
Verhältniszahl oder Proportionalität bestimmte Arbeitermassen unter
bestimmte Funktionen subsumiert, treiben Zufall und Willkür ihr buntes
Spiel in der Verteilung der Warenproduzenten und ihrer Produktionsmittel
unter die verschiednen gesellschaftlichen Arbeitszweige. Zwar suchen
sich die verschiednen Produktionssphären beständig ins Gleichgewicht zu
setzen, indem einerseits jeder Warenproduzent einen Gebrauchswert
produzieren, \num{377} also ein besondres
gesellschaftliches Bedürfnis befriedigen muß, der Umfang dieser
Bedürfnisse aber quantitativ verschieden ist und ein innres Band die
verschiednen Bedürfnismassen zu einem naturwüchsigen System verkettet;
indem andrerseits das Wertgesetz der Waren bestimmt, wieviel die
Gesellschaft von ihrer ganzen disponiblen Arbeitszeit auf die Produktion
jeder besondren Warenart verausgaben kann. Aber diese beständige Tendenz
der verschiednen Produktionssphären, sich ins Gleichgewicht zu setzen,
betätigt sich nur als Reaktion gegen die beständige Aufhebung dieses
Gleichgewichts. Die bei der Teilung der Arbeit im Innern der Werkstatt a
priori und planmäßig befolgte Regel wirkt bei der Teilung der Arbeit im
Innern der Gesellschaft nur a posteriori als innre, stumme, im
Barometerwechsel der Marktpreise wahrnehmbare, die regellose Willkür der
Warenproduzenten überwältigende Naturnotwendigkeit. Die manufakturmäßige
Teilung der Arbeit unterstellt die unbedingte Autorität des Kapitalisten
über Menschen, die bloße Glieder eines ihm gehörigen Gesamtmechanismus
bilden; die gesellschaftliche Teilung der Arbeit stellt unabhängige
Warenproduzenten einander gegenüber, die keine andre Autorität
anerkennen als die der Konkurrenz, den Zwang, den der Druck ihrer
wechselseitigen Interessen auf sie ausübt, wie auch im Tierreich das
bellum omnium contra omnes die Existenzbedingungen aller Arten mehr oder
minder erhält. Dasselbe bürgerliche Bewußtsein, das die manufakturmäßige
Teilung der Arbeit, die lebenslängliche Annexation des Arbeiters an eine
Detailverrichtung und die unbedingte Unterordnung der Teilarbeiter unter
das Kapital als eine Organisation der Arbeit feiert, welche ihre
Produktivkraft steigre, denunziert daher ebenso laut jede bewußte
gesellschaftliche Kontrolle und Reglung des gesellschaftliche
Produktionsprozesses als einen Eingriff in die unveretzlichen
Eigentumsrechte, Freiheit und sich selbst bestimmende ``Genialität'' des
individuellen Kapitalisten. Es ist sehr charakteristisch, daß die
begeisterten Apologeten des Fabriksystems nichts Ärgres gegen jede
allgemeine Organisation der gesellschaftlichen Arbeit zu sagen wissen,
als daß sie die ganze Gesellschaft in eine Fabrik verwandeln würde.

Wenn die Anarchie der gesellschaftlichen und die Despotie der
manufakturmäßigen Arbeitsteilung einander in der Gesellschaft der
kapitalistischen Produktionsweise bedingen, bieten dagegen frühere
Gesellschaftsformen, worin die Besonderung der Gewerbe sich naturwüchsig
entwickelt, dann kristallisiert und endlich gesetzlich befestigt hat,
einerseits das Bild einer plan- und autoritätsmäßigen Organisation der
gesellschaftlichen Arbeit, während sie anderseits die Teilung der Arbeit
innerhalb der \num{378} Werkstatt ganz ausschließen
oder nur auf einem Zwergmaßstab oder nur sporadisch und zufällig
entwickeln.(59)

Jene uraltertümlichen, kleinen indischen Gemeinwesen z.B., die zum Teil
noch fortexistieren, beruhn auf gemeinschaftlichem Besitz des Grund und
Bodens, auf unmittelbarer Verbindung von Agrikultur und Handwerk und auf
einer festen Teilung der Arbeit, die bei Anlage neuer Gemeinwesen als
gegebner Plan und Grundriß dient. Sie bilden sich selbst genügende
Produktionsganze, deren Produktionsgebiet von 100 bis auf einige 1.000
Acres wechselt. Die Hauptmasse der Produkte wird für den unmittelbaren
Selbstbedarf der Gemeinde produziert, nicht als Ware, und die Produktion
selbst ist daher unabhängig von der durch Warenaustausch vermittelten
Teilung der Arbeit im großen und ganzen der indischen Gesellschaft. Nur
der Überschuß der Produkte verwandelt sich in Ware, zum Teil selbst
wieder erst in der Hand des Staats, dem ein bestimmtes Quantum seit
undenklichen Zeiten als Naturalrente zufließt. Verschiedne Teile Indiens
besitzen verschiedne Formen des Gemeinwesens. In der einfachsten Form
bebaut die Gemeinde das Land gemeinschaftlich und verteilt seine
Produkte unter ihre Glieder, während jede Familie Spinnen, Weben usw.
als häusliches Nebengewerb treibt. Neben dieser gleichartig
beschäftigten Masse finden wir den ``Haupteinwohner'', Richter, Polizei
und Steuereinnehmer in einer Person; den Buchhalter, der die Rechnung
über den Ackerbau führt und alles darauf Bezügliche katastriert und
registriert; einen Beamten, der Verbrecher verfolgt und fremde Reisende
beschützt und von einem Dorf zum andren geleitet; den Grenzmann, der die
Grenzen der Gemeinde gegen die Nachbargemeinden bewacht; den
Wasseraufseher, der das Wasser aus den gemeinschaftlichen
Wasserbehältern zu Ackerbauzwecken verteilt; den Braminen, der die
Funktionen des religiösen Kultus verrichtet; den Schulmeister, der die
Gemeindekinder im Sand schreiben und lesen lehrt; den Kalenderbraminen,
der als Astrolog die Zeiten für Saat, Ernte und die guten und bösen
Stunden für alle besondren Ackerbauarbeiten angibt; einen Schmied und
\num{379} einen Zimmermann, welche alle
Ackerbauwerkzeuge verfertigen und ausbessern; den Töpfer, der alle
Gefäße für das Dorf macht; den Barbier, den Wäscher für die Reinigung
der Kleider, den Silberschmied, hier und da den Poeten, der in einigen
Gemeinden den Silberschmied, in andren den Schulmeister ersetzt. Dies
Dutzend Personen wird auf Kosten der ganzen Gemeinde erhalten. Wächst
die Bevölkerung, so wird eine neue Gemeinde nach dem Muster der alten
auf unbebautem Boden angesiedelt. Der Gemeindemechanismus zeigt
planmäßige Teilung der Arbeit, aber ihre manufakturmäßige Teilung ist
unmöglich, indem der Markt für Schmied, Zimmermann usw. unverändert
bleibt und höchstens, je nach dem Größenunterschied der Dörfer, statt
eines Schmieds, Töpfers usw. ihrer zwei oder drei vorkommen.(60) Das
Gesetz, das die Teilung der Gemeindearbeit regelt, wirkt hier mit der
unverbrüchlichen Autorität eines Naturgesetzes, während jeder besondre
Handwerker, wie Schmied usw., nach überlieferter Art, aber selbständig
und ohne Anerkennung irgendeiner Autorität in seiner Werkstatt, alle zu
seinem Fach gehörigen Operationen verrichtet. Der einfache produktive
Organismus dieser selbstgenügenden Gemeinwesen, die sich beständig in
derselben Form reproduzieren und, wenn zufällig zerstört, an demselben
Ort, mit demselben Namen, wieder aufbauen (61), liefert den Schlüssel
zum Geheimnis der Unveränderlichkeit asiatischer Gesellschaften, so
auffallend kontrastiert durch die beständige Auflösung und Neubildung
asiatischer Staaten und rastlosen Dynastenwechsel. Die Struktur der
ökonomischen Grundelemente der Gesellschaft bleibt von den Stürmen der
politischen Wolkenregion unberührt.

Die Zunftgesetze, wie schon früher bemerkt, verhinderten planmäßig,
\num{380} durch äußerste Beschränkung der
Gesellenzahl, die ein einzelner Zunftmeister beschäftigen durfte, seine
Verwandlung in einen Kapitalisten. Ebenso konnte er Gesellen nur
beschäftigen in dem ausschließlichen Handwerk, worin er selbst Meister
war. Die Zunft wehrte eifersüchtig jeden Übergriff des Kaufmannskapitals
ab, der einzig freien Form des Kapitals, die ihr gegenüberstand. Der
Kaufmann konnte alle Waren kaufen, nur nicht die Arbeit als Ware. Er war
nur geduldet als Verleger der Handwerksprodukte. Riefen äußere Umstände
eine fortschreitende Teilung der Arbeit hervor, so zerspalteten sich
bestehende Zünfte in Unterarten oder lagerten sich neue Zünfte neben die
alten hin, jedoch ohne Zusammenfassung verschiedner Handwerke in einer
Werkstatt. Die Zunftorganisation, sosehr ihre Besondrung, Isolierung und
Ausbildung der Gewerbe zu den materiellen Existenzbedingungen der
Manufakturperiode gehören, schloß daher die manufakturmäßige Teilung der
Arbeit aus. Im großen und ganzen blieben der Arbeiter und seine
Produktionsmittel miteinander verbunden wie die Schnecke mit dem
Schneckenhaus, und so fehlte die erste Grundlage der Manufaktur, die
Verselbständigung der Produktionsmittel als Kapital gegenüber dem
Arbeiter.

Während die Teilung der Arbeit im Ganzen einer Gesellschaft, ob
vermittelt oder unvermittelt durch den Warenaustausch, den
verschiedenartigsten ökonomischen Gesellschaftsformationen angehört, ist
die manufakturmäßige Teilung der Arbeit eine ganz spezifische Schöpfung
der kapitalistischen Produktionsweise.

{%
\section{5. Der kapitalistische Charakter der
Manufaktur}\label{der-kapitalistische-charakter-der-manufaktur}}

Eine größere Arbeiteranzahl unter dem Kommando desselben Kapitals bildet
den naturwüchsigen Ausgangspunkt, wie der Kooperation überhaupt, so der
Manufaktur. Umgekehrt entwickelt die manufakturmäßige Teilung der Arbeit
das Wachstum der angewandten Arbeiterzahl zur technischen Notwendigkeit.
Das Arbeiterminimum, das ein einzelner Kapitalist anwenden muß, ist ihm
jetzt durch die vorhandne Teilung der Arbeit vorgeschrieben. Andrerseits
sind die Vorteile weitrer Teilung bedingt durch weitere Vermehrung der
Arbeiteranzahl, die nur noch in Vielfachen ausführbar. Mit dem variablen
muß aber auch der konstante Bestandteil des Kapitals wachsen, neben dem
Umfang der gemeinsamen Produktionsbedingungen, wie Baulichkeiten, Öfen
usw., namentlich auch und viel rascher als die Arbeiteranzahl, das
Rohmaterial. Seine Masse, verzehrt in \num{381}
gegebner Zeit durch gegebnes Arbeitsquantum, nimmt in demselben
Verhältnis zu wie die Produktivkraft der Arbeit infolge ihrer Teilung.
Wachsender Minimalumfang von Kapital in der Hand der einzelnen
Kapitalisten oder wachsende Verwandlung der gesellschaftlichen
Lebensmittel und Produktionsmittel in Kapital ist also ein aus dem
technischen Charakter der Manufaktur entspringendes Gesetz.(62)

Wie in der einfachen Kooperation ist in der Manufaktur der
funktionierende Arbeitskörper eine Existenzform des Kapitals. Der aus
vielen individuellen Teilarbeitern zusammengesetzte gesellschaftliche
Produktionsmechanismus gehört dem Kapitalisten. Die aus der Kombination
der Arbeiten entspringende Produktivkraft erscheint daher als
Produktivkraft des Kapitals. Die eigentliche Manufaktur unterwirft nicht
nur den früher selbständigen Arbeiter dem Kommando und der Disziplin des
Kapitals, sondern schafft überdem eine hierarchische Gliederung unter
den Arbeitern selbst. Während die einfache Kooperation die Arbeitsweise
der einzelnen im großen und ganzen unverändert läßt, revolutioniert die
Manufaktur sie von Grund aus und ergreift die individuelle Arbeitskraft
an ihrer Wurzel. Sie verkrüppelt den Arbeiter in eine Abnormität, indem
sie sein Detailgeschick treibhausmäßig fördert durch Unterdrückung einer
Welt von produktiven Trieben und Anlagen, wie man in den
La-Plata-Staaten ein ganzes Tier abschlachtet, um sein Fell oder seinen
Talg zu erbeuten. Die besondren Teilarbeiten werden nicht nur unter
verschiedne Individuen verteilt, sondern das Individuum selbst wird
geteilt, in das automatische Triebwerk einer Teilarbeit verwandelt (63)
und die abgeschmackte Fabel des Menenius Agrippa verwirklicht, die einen
Menschen als bloßes Fragment \num{382} seines eignen
Körpers darstellt.(64) Wenn der Arbeiter ursprünglich seine Arbeitskraft
an das Kapital verkauft, weil ihm die materiellen Mittel zur Produktion
einer Ware fehlen, versagt jetzt seine individuelle Arbeitskraft selbst
ihren Dienst, sobald sie nicht an das Kapital verkauft wird. Sie
funktioniert nur noch in einem Zusammenhang, der erst nach ihrem Verkauf
existiert, in der Werkstatt des Kapitalisten. Seiner natürlichen
Beschaffenheit nach verunfähigt, etwas Selbständiges zu machen,
entwickelt der Manufakturarbeiter produktive Tätigkeit nur noch als
Zubehör zur Werkstatt des Kapitalisten.(65) Wie dem auserwählten Volk
auf der Stirn geschrieben stand, daß es das Eigentum Jehovas, so drückt
die Teilung der Arbeit dem Manufakturarbeiter einen Stempel auf, der ihn
zum Eigentum des Kapitals brandmarkt.

Die Kenntnisse, die Einsicht und der Wille, die der selbständige Bauer
oder Handwerker, wenn auch auf kleinem Maßstab, entwickelt, wie der
Wilde alle Kunst des Kriegs als persönliche List ausübt, sind jetzt nur
noch für das Ganze der Werkstatt erheischt. Die geistigen Potenzen der
Produktion erweitern ihren Maßstab auf der einen Seite, weil sie auf
vielen Seiten verschwinden. Was die Teilarbeiter verlieren, konzentriert
sich ihnen gegenüber im Kapital.(66) Es ist ein Produkt der
manufakturmäßigen Teilung der Arbeit, ihnen die geistigen Potenzen des
materiellen Produktionsprozesses als fremdes Eigentum und sie
beherrschende Macht gegenüberzustellen. Dieser Scheidungsprozeß beginnt
in der einfachen Kooperation, wo der Kapitalist den einzelnen Arbeitern
gegenüber die Einheit und den Willen des gesellschaftlichen
Arbeitskörpers vertritt. Er entwickelt sich in der Manufaktur, die den
Arbeiter zum Teilarbeiter verstümmelt. Er vollendet sich in der großen
Industrie, welche die Wissenschaft als selbständige Produktionspotenz
von der Arbeit trennt und in den Dienst des Kapitals preßt.(67)

\num{383} In der Manufaktur ist die Bereicherung des
Gesamtarbeiters und daher des Kapitals an gesellschaftlicher
Produktivkraft bedingt durch die Verarmung des Arbeiters an
individuellen Produktivkräften.

``Die Unwissenheit ist die Mutter der Industrie wie des Aberglaubens.
Nachdenken und Einbildungskraft sind dem Irrtum unterworfen; aber die
Gewohnheit, den Fuß oder die Hand zu bewegen, hängt weder von dem einen
noch von der andren ab. Manufakturen prosperieren also da am meisten, wo
man am meisten sich des Geistes entschlägt, in der Art, daß die
Werkstatt als eine Maschine betrachtet werden kann, deren Teile Menschen
sind.''(68)

In der Tat wandten einige Manufakturen in der Mitte des 18. Jahrhunderts
für gewisse einfache Operationen, welche aber Fabrikgeheimnisse
bildeten, mit Vorliebe halbe Idioten an.(69)

``Der Geist der großen Mehrzahl der Menschen'', sagt A. Smith,
``entwickelt sich notwendig aus und an ihren Alltagsverrichtungen. Ein
Mensch, der sein ganzes Leben in der Verrichtung weniger einfacher
Operationen verausgabt \ldots{} hat keine Gelegenheit, seinen Verstand
zu üben \ldots{} Er wird im allgemeinen so stupid und unwissend, wie es
für eine menschliche Kreatur möglich ist.''

Nachdem Smith den Stumpfsinn des Teilarbeiters geschildert, fährt er
fort:

``Die Einförmigkeit seines stationären Lebens verdirbt natürlich auch
den Mut seines Geistes \ldots{} Sie zerstört selbst die Energie seines
Körpers und verunfähigt ihn, seine Kraft schwunghaft und ausdauernd
anzuwenden, außer in der Detailbeschäftigung, wozu er herangezogen ist.
Sein Geschick in seinem besondren Gewerke scheint so erworben auf Kosten
seiner intellektuellen, sozialen und kriegerischen Tugenden. Aber in
jeder industriellen und zivilisierten Gesellschaft ist dies der Zustand,
worin der arbeitende Arme (the labouring poor), d.h. die große Masse des
Volks notwendig verfallen muß.''(70)

\num{384} Um die aus der Teilung der Arbeit
entspringende völlige Verkümmerung der Volksmasse zu verhindern,
empfiehlt A. Smith Volksunterricht von Staats wegen, wenn auch in
vorsichtig homöopathischen Dosen. Konsequent polemisiert dagegen sein
französischer Übersetzer und Kommentator, G. Garnier, der sich unter dem
ersten französischen Kaisertum naturgemäß zum Senator entpuppte.
Volksunterricht verstoße wider die ersten Gesetze der Teilung der Arbeit
und mit demselben ``proskribiere man unser ganzes Gesellschaftssystem''.

``Wie alle andren Teilungen der Arbeit'', sagte er, ``wird die zwischen
Handarbeit und Verstandesarbeit (71) ausgesprochner und entschiedner im
Maße, wie die Gesellschaft'' (er wendet richtig diesen Ausdruck an für
das Kapital, das Grundeigentum und ihren Staat) ``reicher wird. Gleich
jeder andren ist diese Teilung der Arbeit eine Wirkung vergangner und
eine Ursache künftiger Fortschritte \ldots{} Darf die Regierung denn
dieser Teilung der Arbeit entgegenwirken und sie in ihrem naturgemäßen
Gang aufhalten? Darf sie einen Teil der Staatseinnahme zum Versuch
verwenden, zwei Klassen von Arbeit, die ihre Teilung und Trennung
erstreben, zu verwirren und zu vermischen?''(72)

Eine gewisse geistige und körperliche Verkrüppelung ist unzertrennlich
selbst von der Teilung der Arbeit im ganzen und großen der Gesellschaft.
Da aber die Manufakturperiode diese gesellschaftliche Zerspaltung der
Arbeitszweige viel weiter führt, andrerseits erst mit der ihr
eigentümlichen Teilung das Individuum an seiner Lebenswurzel ergreift,
liefert sie auch zuerst das Material und den Anstoß zur industriellen
Pathologie.(73)

\num{385} ``Einen Menschen unterabteilen, heißt ihn
hinrichten, wenn er das Todesurteil verdient, ihn meuchelmorden, wenn er
es nicht verdient. Die Unterabteilung der Arbeit ist der Meuchelmord
eines Volks.''(74)

Die auf Teilung der Arbeit beruhende Kooperation oder die Manufaktur ist
in ihren Anfängen ein naturwüchsiges Gebild. Sobald sie einige
Konsistenz und Breite des Daseins gewonnen, wird sie zur bewußten,
planmäßigen und systematischen Form der kapitalistischen
Produktionsweise. Die Geschichte der eigentlichen Manufaktur zeigt, wie
die ihr eigentümliche Teilung der Arbeit zunächst erfahrungsmäßig,
gleichsam hinter dem Rücken der handelnden Personen, die sachgemäßen
Formen gewinnt, dann aber, gleich dem zünftigen Handwerke, die einmal
gefundne Form traditionell festzuhalten strebt und in einzelnen Fällen
jahrhundertlang festhält. Ändert sich diese Form, so, außer in
Nebendingen, immer nur infolge einer Revolution der Arbeitsinstrumente.
Die moderne Manufaktur - ich spreche hier nicht von der auf Maschinerie
beruhenden großen Industrie - findet entweder, wie z.B. die
Kleidermanufaktur, in den großen Städten, wo sie entsteht, die disjecta
membra poetae bereits fertig vor und hat sie nur aus ihrer Zerstreuung
zu sammeln, oder das Prinzip der Teilung liegt auf flacher Hand, indem
einfach die verschiednen Verrichtungen der handwerksmäßigen Produktion
(z.B. beim Buchbinden) besondren Arbeitern ausschließlich angeeignet
werden. Es kostet noch keine Woche Erfahrung, in solchen Fällen die
Verhältniszahl zwischen den für jede Funktion nötigen Händen zu
finden.(75)

\num{386} Die manufakturmäßige Teilung der Arbeit
schafft durch Analyse der handwerksmäßigen Tätigkeit, Spezifizierung der
Arbeitsinstrumente, Bildung der Teilarbeiter, ihre Gruppierung und
Kombination in einem Gesamtmechanismus, die qualitative Gliederung und
quantitative Proportionalität gesellschaftlicher Produktionsprozesse,
also eine bestimmte Organisation gesellschaftlicher Arbeit und
entwickelt damit zugleich neue, gesellschaftliche Produktivkraft der
Arbeit. Als spezifisch kapitalistische Form des gesellschaftlichen
Produktionsprozesses - und auf den vorgefundnen Grundlagen konnte sie
sich nicht anders als in der kapitalistischen Form entwickeln - ist sie
nur eine besondre Methode, relativen Mehrwert zu erzeugen oder die
Selbstverwertung des Kapitals - was man gesellschaftlichen Reichtum,
``Wealth of Nations'' usw. nennt - auf Kosten der Arbeiter zu erhöhn.
Sie entwickelt die gesellschaftliche Produktivkraft der Arbeit nicht nur
für den Kapitalisten, statt für den Arbeiter, sondern durch die
Verkrüpplung des individuellen Arbeiters. Sie produziert neue
Bedingungen der Herrschaft des Kapitals über die Arbeit. Wenn sie daher
einerseits als historischer Fortschritt und notwendiges
Entwicklungsmoment im ökonomischen Bildungsprozeß der Gesellschaft
erscheint, so andrerseits als ein Mittel zivilisierter und raffinierter
Exploitation.

Die politische Ökonomie, die als eigne Wissenschaft erst in der
Manufakturperiode aufkommt, betrachtet die gesellschaftliche Teilung der
Arbeit überhaupt nur vom Standpunkt der manufakturmäßigen Teilung der
Arbeit (76), als Mittel, mit demselben Quantum Arbeit mehr Ware zu
produzieren, daher die Waren zu verwohlfeilern und die Akkumulation des
Kapitals zu beschleunigen. Im strengsten Gegensatz zu dieser
Akzentuierung der Quantität und des Tauschwerts halten sich die
Schriftsteller des klassischen Altertums ausschließlich an Qualität und
Gebrauchswert.(77) In- \num{387} folge der Scheidung
der gesellschaftlichen Produktionszweige werden die Waren besser
gemacht, die verschiednen Triebe und Talente der Menschen wählen sich
entsprechende Wirkungssphären (78), und ohne Beschränkung ist nirgendwo
Bedeutendes zu leisten.(79) Also Produkt und Produzent werden verbessert
durch die Teilung der Arbeit. Wird gelegentlich auch das Wachstum der
Produktenmasse erwähnt, so nur mit Bezug auf die größre Fülle des
Gebrauchswerts. Es wird mit keiner Silbe des Tauschwerts, der
Verwohlfeilerung der Waren gedacht. Dieser Standpunkt des Gebrauchswerts
herrscht sowohl bei Plato (80), der die Teilung der Arbeit als
\num{388} Grundlage der gesellschaftlichen Scheidung
der Stände behandelt, als bei Xenophon (81), der mit seinem
charakteristisch bürgerlichen Instinkt schon der Teilung der Arbeit
innerhalb einer Werkstatt näher rückt. Platos Republik, soweit in ihr
die Teilung der Arbeit als das gestaltende Prinzip des Staats entwickelt
wird, ist nur atheniensische Idealisierung des ägyptischen Kastenwesens,
wie Ägypten als industrielles Musterland auch andren seiner Zeitgenossen
gilt, z.B. dem Isokrates (82), und diese Be-
\num{389} deutung selbst noch für die Griechen der
römischen Kaiserzeit behielt.(83)

Während der eigentlichen Manufakturperiode, d.h. der Periode, worin die
Manufaktur die herrschende Form der kapitalistischen Produktionsweise,
stößt die volle Ausführung ihrer eignen Tendenzen auf vielseitige
Hindernisse. Obgleich sie, wie wir sahen, neben der hierarchischen
Gliederung der Arbeiter eine einfache Scheidung zwischen geschickten und
ungeschickten Arbeitern schafft, bleibt die Zahl der letztren durch den
überwiegenden Einfluß der erstren sehr beschränkt. Obgleich sie die
Sonderoperationen dem verschiednen Grad von Reife, Kraft und Entwicklung
ihrer lebendigen Arbeitsorgane anpaßt und daher zu produktiver
Ausbeutung von Weibern und Kindern drängt, scheitert diese Tendenz im
großen und ganzen an den Gewohnheiten und dem Widerstand der männlichen
Arbeiter. Obgleich die Zersetzung der handwerksmäßigen Tätigkeit die
Bildungskosten und daher den Wert der Arbeiter senkt, bleibt für
schwierigere Detailarbeit eine längre Erlernungszeit nötig und wird auch
da, wo sie vom Überfluß, eifersüchtig von den Arbeitern
aufrechterhalten. Wir finden z.B. in England die laws of apprenticeship
mit ihrer siebenjährigen Lernzeit bis zum Ende der Manufakturperiode in
Vollkraft und erst von der großen Industrie über Haufen geworfen. Da das
Handwerksgeschick die Grundlage der Manufaktur bleibt und der in ihr
funktionierende Gesamtmechanismus kein von den Arbeitern selbst
unabhängiges objektives Skelett besitzt, ringt das Kapital beständig mit
der Insubordination der Arbeiter.

``Die Schwäche der menschlichen Natur'', ruft Freund Ure aus, ``ist so
groß, daß der Arbeiter, je geschickter, desto eigenwilliger und
schwieriger zu behandeln wird und folglich dem Gesamtmechanismus durch
seine rappelköpfigen Launen schweren Schaden zufügt.''(84)

\num{390} Durch die ganze Manufakturperiode läuft
daher die Klage über den Disziplinmangel der Arbeiter.(85) Und hätten
wir nicht die Zeugnisse gleichzeitiger Schriftsteller, die einfachen
Tatsachen, daß es vom 16. Jahrhundert bis zur Epoche der großen
Industrie dem Kapital mißlingt, sich der ganzen disponiblen Arbeitszeit
der Manufakturarbeiter zu bemächtigen, daß die Manufakturen kurzlebig
sind und mit der Ein- oder Auswandrung der Arbeiter ihren Sitz in dem
einen Land verlassen und in dem andren aufschlagen, würden Bibliotheken
sprechen. ``Ordnung muß auf die eine oder die andre Weise gestiftet
werden'', ruft 1770 der wiederholt zitierte Verfasser des ``Essay on
Trade and Commerce''. Ordnung, hallt es 66 Jahre später zurück aus dem
Mund des Dr.~Andrew Ure, ``Ordnung'' fehlte in der auf ``dem
scholastischen Dogma der Arbeit'' beruhenden Manufaktur, und ``Arkwright
schuf die Ordnung''.

Zugleich konnte die Manufaktur die gesellschaftliche Produktion weder in
ihrem ganzen Umfang ergreifen noch in ihrer Tiefe umwälzen. Sie gipfelte
als ökonomisches Kunstwerk auf der breiten Grundlage des städtischen
Handwerks und der ländlich häuslichen Industrie. Ihre eigne enge
technische Basis trat auf einem gewissen Entwicklungsgrad mit den von
ihr selbst geschaffnen Produktionsbedürfnissen in Widerspruch.

Eins ihrer vollendetsten Gebilde war die Werkstatt zur Produktion der
Arbeitsinstrumente selbst, und namentlich auch der bereits angewandten
komplizierteren mechanischen Apparate.

``Ein solches Atelier'', sagt Ure, ``bot dem Auge die Teilung der Arbeit
in ihren mannigfachen Abstufungen. Bohrer, Meißel, Drechselbank hatten
jede ihre eignen Arbeiter, hierarchisch gegliedert nach dem Grad ihrer
Geschicklichkeit.''

Dies Produkt der manufakturmäßigen Teilung der Arbeit produzierte
seinerseits - Maschinen. Sie heben die handwerksmäßige Tätigkeit als das
regelnde Prinzip der gesellschaftlichen Produktion auf. So wird
einerseits der technische Grund der lebenslangen Annexation des
Arbeiters an eine Teilfunktion weggeräumt. Andrerseits fallen die
Schranken, welche dasselbe Prinzip der Herrschaft des Kapitals noch
auferlegte.

{%
\section{Fußnoten}\label{fuuxdfnoten}}

\begin{enumerate}
\def\labelenumi{(\arabic{enumi})}
\setcounter{enumi}{25}
\item
  Um ein mehr modernes Beispiel dieser Bildungsart der Manufaktur
  anzuführen, folgendes Zitat. Die Seidenspinnerei und Weberei von Lyon
  und Nîmes ``ist ganz patriarchalisch; sie beschäftigt viele Frauen und
  Kinder, aber ohne sie zu übermüden oder zugrunde zu richten; sie läßt
  sie in ihren schönen Tälern der Drôme, des Var, der Isère und von
  Vaucluse, um dort Seidenraupen zu züchten, und ihre Kokons
  abzuwickeln; sie wird niemals zu einem regelrechten Fabrikbetrieb. Um
  trotzdem in so hohen Maße angewandt zu werden \ldots{} nimmt hier das
  Prinzip der Arbeitsteilung eine besondere Eigenart an. Es gibt zwar
  Hasplerinnen, Seidenzwirner, Färber, Kettenschlichter, ferner Weber;
  aber sie sind nicht in derselben Werkstatt vereinigt, nicht von
  demselben Meister abhängig; alle sind sie unabhängig.'' (A. Blanqui,
  ``Cours d'Écon. Industrielle'', Recueilli par A. Blaise, Paris
  1838-1839, p.~79.) Seit Blanqui dies schrieb, sind die verschiednen
  unabhängigen Arbeiter zu Teil in Fabriken vereinigt worden. \{Zur 4.
  Aufl. - Und seit Marx obiges schrieb, hat der Kraftstuhl sich in
  diesen Fabriken eingebürgert und verdrängt rasch den Handwebstuhl. Die
  Krefelder Seidenindustrie weiß ebenfalls ein Lied davon zu singen. -
  F. E.\} \textless{}=
\item
  ``Je mehr eine Arbeit von großer Mannigfaltigkeit gegliedert und
  verschiedenen Teilarbeitern zugewiesen wird, um so mehr muß sie
  notwendigerweise besser und schneller ausgeführt werden, mit weniger
  Verlust an Zeit und Arbeit.'' (``The Advantages of the East India
  Trade'', Lond. 1720, p.~71.) \textless{}=
\item
  ``Leicht von der Hand gehende Arbeit ist überlieferte
  Geschicklichkeit.'' (Th. Hodgskin, Popular Political Economy, p.~48.)
  \textless{}=
\item
  ``Auch die Künste sind \ldots{} in Ägypten zu dem gehörigen Grad von
  Vollkommenheit gediehn. Denn in diesem Lande allein dürfen die
  Handwerker durchaus nicht in die Geschäfte einer andren Bürgerklasse
  eingreifen, sondern bloß den nach dem Gesetz ihrem Stamme erblich
  zugehörigen Beruf treiben \ldots{} Bei andren Völkern findet man, daß
  die Gewerbsleute ihre Aufmerksamkeit auf zu viele Gegenstände
  verteilen \ldots{} Bald versuchen sie es mit dem Landbau, bald lassen
  sie sich in Handelsgeschäfte ein, bald befassen sie sich mit zwei oder
  drei Künsten zugleich. In Freistaaten laufen sie meist in die
  Volksversammlungen \ldots{} In Ägypten dagegen verfällt jeder
  Handwerker in schwere Strafen, wenn er sich in Staatsgeschäfte mischt
  oder mehrere Künste zugleich treibt. So kann nichts ihren Berufsfleiß
  stören \ldots{} Zudem, wie sie von ihren Vorfahren viele Regeln haben,
  sind sie eifrig darauf bedacht, noch neue Vorteile aufzufinden.''
  (Diodorus Siculus: ``Historische Bibliothek'', I. I, c. 74.)
  \textless{}=
\item
  ``Historical and descriptive Account of Brit. India etc.'' By Hugh
  Murray, James Wilson etc., Edinburgh 1832, v. II, p.~449, 450. Der
  indische Webstuhl ist hochschäftig, d.h., die Kette ist vertikal
  aufgespannt. \textless{}=
\item
  Darwin bemerkt in seinem epochemachenden Werk ``Über die Entstehung
  der Arten'' mit Bezug auf die natürlichen Organe der Pflanzen und
  Tiere: ``Solange ein und dasselbe Organ verschiedne Arbeiten zu
  verrichten hat, läßt sich ein Grund für seine Veränderlichkeit
  vielleicht darin finden, daß natürliche Züchtung jede kleine
  Abweichung der Form weniger sorgfältig erhält oder unterdrückt, als
  wenn dasselbe Organ nur zu einem besondren Zwecke allein bestimmt
  wäre. So mögen Messer, welche allerlei Dinge zu schneiden bestimmt
  sind, im ganzen so ziemlich von einerlei Form sein, während ein nur zu
  einerlei Gebrauch bestimmtes Werkzeug für jeden andren Gebrauch auch
  eine andre Form haben muß.'' \textless{}=
\item
  Genf hat im Jahr 1854 80.000 Uhren produziert, noch nicht ein Fünfteil
  der Uhrenproduktion des Kantons Neuchâtel. Chaux-de-Fonds, das man als
  eine einzige Uhrenmanufaktur betrachten kann, liefert allein jährlich
  doppelt soviel als Genf. Von 1850-1861 lieferte Genf 720.000 Uhren.
  Siehe ``Report from Geneva on the Watch Trade'' in ``Reports by H.
  M.'s Secretaries of Embassy and Legation on the Manufactures, Commerce
  etc.'', Nr. 6, 1863. Wenn die Zusammenhangslosigkeit der Prozesse,
  worin die Produktion nur zusammengesetzter Machwerke zerfällt, an und
  für sich die Verwandlung solcher Manufakturen in den Maschinenbetrieb
  der großen Industrie sehr erschwert, kommen bei der Uhr noch zwei
  andre Hindernisse hinzu, die Kleinheit und Delikatesse ihrer Elemente
  und ihr Luxuscharakter, daher ihre Varietät, so daß z.B. in den besten
  Londoner Häusern das ganze Jahr hindurch kaum ein Dutzend Uhren
  gemacht werden, die sich ähnlich sehn. Die Uhrenfabrik von Vacheron \&
  Constantin, die mit Erfolg Maschinerie anwendet, liefert auch
  höchstens 3-4 verschiedne Varietäten von Größe und Form. \textless{}=
\item
  In der Uhrmacherei, diesem klassischen Beispiel der heterogenen
  Manufaktur, kann man sehr genau die oben erwähnte aus der Zersetzung
  der handwerksmäßigen Tätigkeit entspringende Differenzierung und
  Spezialisierung der Arbeitsinstrumente studieren. \textless{}=
\item
  ``Wenn die Menschen so dicht nebeneinander arbeiten, muß der Transport
  notwendigerweise geringer sein.'' (``The Advantages of the East India
  Trade'', p.~106.) \textless{}=
\item
  ``Die Vereinzelung der verschiedenen Produktionsstufen in der
  Manufaktur, die aus der Verwendung von Handarbeit folgt, erhöht die
  Produktionskosten ungeheuer, wobei der Verlust in der Hauptsache durch
  die bloße Beförderung von einem Arbeitsprozeß zum anderen entsteht.''
  (``The Industry of Nations'', Lond. 1855, part II, p.~200.)
  \textless{}=
\item
  ``Sie'' (die Teilung der Arbeit) ``verursacht auch eine Zeitersparnis,
  indem sie die Arbeit in ihre verschiedenen Zweige zerlegt, die alle im
  gleichen Augenblick ausgeführt werden können \ldots{} Durch die
  gleichzeitige Durchführung all der verschiedenen Arbeitsprozesse, die
  ein einzelner getrennt hätte ausführen müssen, wird es z.B. möglich,
  eine Menge Nadeln in derselben Zeit fertigzustellen, in der eine
  einzelne Nadel sonst nur abgeschnitten oder zugespitzt worden wäre.''
  (Dugald Stewart, l.c.p. 319.) \textless{}=
\item
  ``Je mannigfaltiger die Spezialarbeiter in jeder Manufaktur, \ldots{}
  um so ordentlicher und regelmäßiger ist jede Arbeit; diese muß
  notwendig in weniger Zeit getan werden, und die Arbeit muß sich
  vermindern.'' (``The Advantages etc.'', p.~68.) \textless{}=
\item
  Indes erreicht der manufakturmäßige Betrieb dies Resultat in vielen
  Zweigen nur unvollkommen, weil er die allgemeinen chemischen und
  physikalischen Bedingungen des Produktionsprozesses nicht mit
  Sicherheit zu kontrollieren weiß. \textless{}=
\item
  ``Wenn die Erfahrung, je nach der besondren Natur der Produkte jeder
  Manufaktur, sowohl die vorteilhafteste Art, die Fabrikation in
  Teiloperationen zu spalten, als auch die für sie nötige Arbeiterzahl
  kennen gelehrt hat, werden alle Etablissements, die kein exaktes
  Multipel dieser Zahl anwenden, mit mehr Kosten fabrizieren \ldots{}
  Dies ist eine der Ursachen der kolossalen Ausdehnung industrieller
  Etablissements.'' (Ch. Babbage, ``On the Economy of Machinery'', Lond.
  1832, ch.~XXI, p.~172, 173.) \textless{}=
\item
  In England ist der Schmelzofen getrennt vom Glasofen, an dem das Glas
  verarbeitet wird, in Belgien z.B. dient derselbe Ofen zu beiden
  Prozessen. \textless{}=
\item
  Man kann dies unter andren ersehn aus W. Petty, John Bellers, Andrew
  Yarranton, ``The Advantages of the East-India Trade'' und J.
  Vanderlint. \textless{}=
\item
  Noch gegen Ende des 16. Jahrhunderts bedient sich Frankreich der
  Mörser und Siebe zum Pochen und Waschen der Erze. \textless{}=
\item
  Die ganze Entwicklungsgeschichte der Maschinerie läßt sich verfolgen
  an der Geschichte der Getreidemühlen. Die Fabrik heißt im Englischen
  immer noch mill . In deutschen technologischen Schriften aus den
  ersten Dezennien des 19. Jahrhunderts findet man noch den Ausdruck
  Mühle nicht nur für alle mit Naturkräften getriebene Maschinerie,
  sondern selbst für alle Manufakturen, die maschinenartige Apparate
  anwenden. \textless{}=
\item
  Wie man aus dem Vierten Buch dieser Schrift näher sehn wird, hat A.
  Smith keinen einzigen neuen Satz über die Teilung der Arbeit
  aufgestellt. Was ihn aber als den zusammenfassenden politischen
  Ökonomen der Manufakturperiode charakterisiert, ist der Akzent, den er
  auf die Teilung der Arbeiter legt. Die untergeordnete Rolle, die er
  der Maschinerie anweist, rief im Beginn der großen Industrie
  Lauderdales, in einer weiterentwickelten Epoche Ures Polemik hervor.
  A. Smith verwechselt auch die Differenzierung der Instrumente, wobei
  die Teilarbeiter der Manufaktur selbst sehr tätig waren, mit der
  Maschinenerfindung. Es sind nicht die Manufakturarbeiter, sondern
  Gelehrte, Handwerker, selbst Bauern (Brindley) usw., die hier eine
  Rolle spielen. \textless{}=
\item
  ``Indem man das Machwerk in mehrere verschiedne Operationen teilt,
  deren jede verschiedne Grade von Gewandtheit und Kraft erheischt, kann
  der Manufakturherr sich genau das jeder Operation entsprechende
  Quantum von Kraft und Gewandtheit verschaffen. Wäre dagegen das ganze
  Werk von einem Arbeiter zu verrichten, so müßte dasselbe Individuum
  genug Gewandtheit für die delikatesten und genug Kraft für die
  mühseligsten Operationen besitzen.'' (Ch. Babbage, l.c., ch.~XIX.)
  \textless{}=
\item
  Z.B. einseitige Muskelentwicklung, Knochenverkrümmung usw.
  \textless{}=
\item
  Sehr richtig antwortet Herr Wm. Marschall, der general manager einer
  Glasmanufakter, auf die Frage des Untersuchungskommissärs, wie die
  Arbeitsamkeit unter den beschäftigten Jungen aufrechterhalten werde:
  ``Sie können ihre Arbeit gar nicht vernachlässigen; haben sie erst
  einmal zu arbeiten begonnen, so müssen sie auch weitermachen; sie sind
  gradeso wie Teile einer Maschine.'' (``Child. Empl. Comm., Fourth
  Report'', 1865, p.~247.) \textless{}=
\item
  Dr.~Ure in seiner Apotheose der großen Industrie fühlt die
  eigentümlichen Charaktere der Manufaktur schärfer heraus als frühere
  Ökonomen, die nicht sein polemisches Interesse hatten, und selbst als
  seine Zeitgenossen, z.B. Babbage, der ihm zwar überlegen ist als
  Mathematiker und Mechaniker, aber dennoch die große Industrie
  eigentlich nur vom Standpunkt der Manufaktur auffaßt. Ure bemerkt:
  ``Die Aneignung der Arbeiter an jede Sonderoperation bildet das Wesen
  der Verteilung der Arbeiten.'' Andrerseits bezeichnet er diese
  Verteilung als ``Anpassung der Arbeiten an die verschiednen
  individuellen Fähigkeiten'' und charakterisiert endlich das ganze
  Manufaktursystem als ``ein System von Gradationen nach dem Rang der
  Geschicklichkeit'', als ``eine Teilung der Arbeit nach den
  verschiednen Graden des Geschicks'' usw. (Ure, ``Philos. of Manuf.'',
  p.~19-23 passim.) \textless{}=
\item
  ``Jeder Handwerker, der \ldots{} instand gesetzt wurde, sich durch die
  Praxis in einer Einzelverrichtung zu vervollkommnen \ldots{} wurde ein
  billigerer Arbeiter.'' (Ure, l.c.p. 19.) \textless{}=
\item
  ``Die Teilung der Arbeit geht von der Trennung der verschidenartigsten
  Professionen fort bis zu jener Teilung, wo mehrere Arbeiter sich in
  die Anfertigung eines und desselben Produkts teilen, wie in der
  Manufaktur.'' (Storch, ``Cours d'Écon. Pol.'', Pariser Ausgabe, t. I,
  p.~173.) ``Wir begegnen bei den Völkern, die eine gewisse Stufe der
  Zivilisation erreicht haben, drei Arten von Arbeitsteilung: die erste,
  die wir die allgemeine nennen, führt die Scheidung der Produzenten in
  Landwirte, Gewerbetreibende und Kaufleute herbei, sie entspricht den
  drei Hauptzweigen der nationalen Arbeit; die zweite, die man die
  besondere nennen könnte, ist die Teilung jedes Arbeitszweigs in Arten
  \ldots{} die dritte Arbeitsteilung endlich, die man als Teilung der
  Arbeitsverrichtung oder als Arbeitsteilung im eigentlichen Sinne
  bezeichnen sollte, ist diejenige, die sich in den einzelnen Handwerken
  und Berufen herausbildet \ldots{} und in den meisten Manufakturen und
  Werkstätten Fuß faßt.'' (Skarbek, l.c.p. 84, 85.) \textless{}=
\end{enumerate}

(50a) \{Note zur 3. Aufl. - Spätere sehr gründliche Studien der
menschlichen Urzustände führten den Verfasser zum Ergebnis, daß
ursprünglich nicht die Familie sich zum Stamm ausgebildet, sondern
umgekehrt, der Stamm die ursprüngliche naturwüchsige Form der auf
Blutsverwandtschaft beruhenden menschlichen Vergesellschaftung war, so
daß aus der beginnenden Auflösung der Stammesbande erst später die
vielfach verschiednen Formen der Familie sich entwickelten. - F. E.\}
\textless{}=

\begin{enumerate}
\def\labelenumi{(\arabic{enumi})}
\setcounter{enumi}{50}
\item
  Sir James Steuart hat diesen Punkt am besten behandelt. Wie wenig sein
  Werk, welches 10 Jahres vor dem ``Wealth of Nations'' erschien,
  heutzutage bekannt ist, sieht man u.a. daraus, daß die Bewundrer des
  Malthus nicht einmal wissen, daß dieser in der ersten Ausgabe seiner
  Schrift über die ``Population'', vom rein deklamatorischen Teil
  abgesehn, neben den Pfaffen Wallace und Townsend fast nur den Steuart
  abschreibt. \textless{}=
\item
  ``Es gibt eine gewisse Bevölkerungsdichte, die zweckdienlich ist,
  sowohl für den gesellschaftlichen Verkehr als auch für jenes
  Zusammenwirken der Kräfte, durch das der Ertrag der Arbeit gesteigert
  wird.'' (James Mill, l.c.p. 50.) ``Wenn die Zahl der Arbeiter wächst,
  steigt die Produktivkraft der Gesellschaft im gleichen Verhältnis zu
  diesem Wachstum, multipliziert mit der Wirkung der Arbeitsteilung.''
  (Th. Hodgskin, l.c.p. 120.) \textless{}=
\item
  Infolge der großen Baumwollnachfrage seit 1861 wurde in einigen sonst
  zahlreich bevölkerten Distrikten Ostindiens die Baumwollproduktion auf
  Kosten der Reisproduktion ausgedehnt. Es entstand daher partielle
  Hungersnot, weil wegen mangelnder Kommunikationsmittel und daher
  mangelnden physischen Zusammenhangs der Reisausfall in einem Distrikt
  nicht durch Zufuhr aus andren Distrikten ausgeglichen werden konnte.
  \textless{}=
\item
  So bildete die Fabrikation der Weberschiffchen schon während des 17.
  Jahrhunderts einen besondren Industriezweig in Holland. \textless{}=
\item
  ``Ist nicht die Wollmanufaktur Englands in verschiedene Teile oder
  Zweige geschieden, die sich an besonderen Orten festgesetzt haben, wo
  sie allein oder hauptsächlich hergestellt werden; feine Tuche in
  Somersetshire, grobe in Yorkshire, doppelbreite in Exeter, Seide in
  Sudbury, Krepps in Norwich, Halbwollstoffe in Kendal, Decken in
  Whitney usw.!'' (Berkeley, ``The Qerist'', 1750, § 520.) \textless{}=
\item
  A. Ferguson, ``History of Civil Society'', Edinb. 1767, part IV, sect.
  II, p.~285. \textless{}=
\item
  In den eigentlichen Manufakturen, sagt er, scheint die Teilung der
  Arbeit größer, weil ``die in jedem einzelnen Arbeitszweig
  Beschäftigten oft in einem Arbeitshaus zusammen sein und vom
  Beobachter mit einem Blick übersehen werden können. In jenen großen
  Manufakturen (!) dagegen, welche dazu bestimmt sind, die
  Hauptbedürfnisse der großen Masse der Bevölkerung zu befriedigen, sind
  in jedem einzelnen Arbeitszweig so viele Arbeiter beschäftigt, daß man
  sie unmöglich in einem Arbeitshaus zusammenbringen kann \ldots{} die
  Teilung ist nicht annähernd so offensichtlich.'' (A. Smith, ``Wealth
  of Nations'', b. I, ch.~I.) Der berühmte Passus in demselben Kapitel,
  der mit den Worten beginnt: ``Man betrachte die Habe des
  gewöhnlichsten Handwerkers oder Tagelöhners in einem zivilisierten und
  blühenden Lande usw.'' und dann weiter ausmalt, wie zahllos
  mannigfaltige Gewerbe zur Befriedigung der Bedürfnisse eines
  gewöhnlichen Arbeiters zusammenwirken, ist ziemlich wörtlich kopiert
  aus B. de Mandevilles Remarks zu seiner ``Fable of the Bees, or,
  Privte Vices, Publick Benefits.'' (Erste Ausgabe ohne Remarks 1705,
  mit den Remarks 1714.) \textless{}=
\item
  ``Es gibt aber nichts mehr, was man als den natürlichen Lohn der
  Arbeit eines einzelnen bezeichnen könnte. Jeder Arbeiter erzeugt nur
  einen Teil eines Ganzen, und da jeder Teil für sich allein ohne Wert
  oder Nutzen ist, gibt es nichts, was der Arbeiter nehmen und wovon er
  sagen könnte: Das ist mein Erzeugnis, das will ich für mich
  behalten.'' (``Labour defended against the claims of Capital'', Lond.
  1825, p.~25.) Der Verfasser dieser vorzüglichen Schrift ist der früher
  zitierte Th. Hodgskin. \textless{}=
\end{enumerate}

(58a) Note zur 2. Ausgabe. Dieser Unterschied zwischen
gesellschaftlicher und manufakturmäßiger Teilung der Arbeit wurde den
Yankees praktisch illustriert. Eine der während des Bürgerkriegs zu
Washington neu ausgeheckten Steuern war die Akzise von 6\% auf ``alle
industriellen Produkte''. Frage: Was ist ein industrielles Produkt?
Antwort des Gesetzgebers: Ein Ding ist produziert, ``wenn es gemacht
ist'' (when it is made), und es ist gemacht, wenn für den Verkauf
fertig. Nun ein Beispiel aus vielen. Manufakturen zu New York und
Philadelphia hatten früher Regenschirme mit allem Zubehör ``gemacht''.
Da ein Regenschirm aber ein Mixtum compositum ganz heterogener
Bestandteile, wurden letztre nach und nach zu Machwerken unabhängig
voneinander und an verschiednen Orten betriebner Geschäftszweige. Ihre
Teilprodukte gingen nun als selbständige Waren ein in die
Regenschirm-Manufaktur, welche sie nur noch in ein Ganzes zusammensetzt.
Die Yankees haben derartige Artikel ``assembled articles'' (versammelte
Artikel) getauft, was sie namentlich verdienten als Sammelplätze von
Steuern. So ``versammelte'' der Regenschirm erstens 6\% Akzise auf den
Preis jedes seiner Elemente und hinwiederum 6\% auf seinen eignen
Gesamtpreis. \textless{}=

\begin{enumerate}
\def\labelenumi{(\arabic{enumi})}
\setcounter{enumi}{58}
\item
  ``Man kann als allgemeine Regel aufstellen: Je weniger die Autorität
  der Teilung der Arbeit innerhalb der Gesellschaft vorsteht, desto mehr
  entwickelt sich die Arbeitsteilung im Innern der Werkstatt und um so
  mehr ist sie der Autorität eines einzelnen unterworfen. Danach steht
  die Autorität in der Werkstatt und die in der Gesellschaft, in bezug
  auf die Arbeitsteilung, im umgekehrten Verhältnis zueinander.'' (Karl
  Marx, l.c.p. 130, 131 \textless{}Siehe Band 4, S. 151\textgreater{}.)
  \textless{}=
\item
  Lieut. Col. Mark Wilks, ``Historical Sketches on the South of India'',
  Lond. 1810 bis 1817, v. I, p.~118-120. Eine gute Zusammenstellung der
  verschiednen Formen des indischen Gemeinwesens findet man in George
  Campbells ``Modern India'', London 1852. \textless{}=
\item
  ``Unter dieser einfachen Form \ldots{} haben die Einwohner des Landes
  seit unvordenklichen Zeiten gelebt. Die Grenzen der Dorfgebiete wurden
  nur selten geändert; und obgleich die Dörfer wiederholt durch Krieg,
  Hungersnot und Seuchen heimgesucht, ja verwüstet wurden, haben
  derselbe Name, dieselben Grenzen, dieselben Interessen und selbst
  dieselben Familien sich durch Generationen fortgesetzt. Die Einwohner
  ließen sich durch den Zusammenbruch und die Teilung von Königreichen
  nicht anfechten; solange das Dorf ungeteilt bleibt, ist es ihnen
  gleichgültig, an welche Macht es abgetreten wird oder welchem
  Herrscher es zufällt. Seine innere Wirtschaft bleibt unverändert.''
  (Th. Stamfort Raffles, late Lieut. Gov.~of Java, ``The History of
  Java'', Lond. 1817, v. I, p.~285.) \textless{}=
\item
  ``Es genügt nicht, daß zur Unterabteilung der Handwerke nötig
  Kapital'' (sollte heißen, die dazu nötigen Lebens- und
  Produktionsmittel) ``sich in der Gesellschaft vorhanden vorfinde; es
  ist außerdem nötig, daß es in den Händen der Unternehmer in
  hinreichend beträchtlichen Massen akkumuliert sei, um sie zur Arbeit
  auf großer Stufenleiter zu befähigen \ldots{} Je mehr die Teilung
  zunimmt, erheischt die beständige Beschäftigung einer selben Zahl von
  Arbeitern immer beträchtlicheres Kapital in Werkzeugen, Rohstoffen
  usw.'' (Storch, ``Cours d'Écon. Polit.'', Pariser Ausg., t. I, p.~250,
  251.) ``Die Konzentration der Produktionsinstrumente und die
  Arbeitsteilung sind ebenso untrennbar voneinander wie auf dem Gebiete
  der Politik die Zentralisation der öffentlichen Gewalten und die
  Teilung der Privatinteressen.'' (Karl Marx, l.c.p. 134
  \textless{}Siehe Band, S.153\textgreater{}.) \textless{}=
\item
  Dugald Stewart nennt die Manufakturarbeiter ``lebende Automaten
  \ldots{} , die für Teilarbeiten verwandt werden''. (l.c.p. 318.)
  \textless{}=
\item
  Bei den Korallen bildet jedes Individuum in der Tat den Magen für die
  ganze Gruppe. Es führt ihr aber Nahrungsstoff zu, statt wie der
  römische Patrizier ihn wegzuführen. \textless{}=
\item
  ``Der Arbeiter, der ein ganzes Handwerk beherrscht, kann überall
  arbeiten und seinen Unterhalt finden: der andere'' (der
  Manufakturarbeiter) ``ist nur noch ein Zubehör und besitzt, von seinen
  Arbeitskollegen getrennt, weder Befähigung noch Unabhängigkeit und ist
  deshalb gezwungen, das Gesetz anzunehmen, das man für richtig hält,
  ihm aufzuerlegen.'' (Storch, l.c., édit. Petersb. 1815, t. I, p.~204.)
  \textless{}=
\item
  A. Ferguson, l.c.p. 281: ``Der eine mag gewonnen haben, was der andere
  verloren hat.'' \textless{}=
\item
  ``Der Mann des Wissens und der produktive Arbeiter sind weit
  voneinander getrennt, und die Wissenschaft, statt in der Hand des
  Arbeiters seine eignen Produktivkräfte für ihn selbst zu vermehren,
  hat sich fast überall ihm gegenübergestellt \ldots{} Kenntnis wird ein
  Instrument, fähig, von der Arbeit getrennt und ihr entgegengesetzt zu
  werden.'' (W. Thompson, ``An Inquiry into the Principles of the
  Distribution of Wealth'', London 1824, p.~274.) \textless{}=
\item
  A. Ferguson, l.c.p. 280. \textless{}=
\item
  J. D. Tuckett, ``A History of the Past and Present State of the
  Labouring Population'', London 1846, v. I, p.~148. \textless{}=
\item
  A. Smith, ``Wealth of Nations'', b. V, ch.~I, art. II. Als Schüler A.
  Fergusons, der die nachteiligen Folgen der Teilung der Arbeit
  entwickelt hatte, war A. Smith über diesen Punkt durchaus klar. Im
  Eingang seines Werks, wo die Teilung der Arbeit exprofesso gefeiert
  wird, deutet er sie nur vorübergehend als Quelle der
  gesellschaftlichen Ungleichheiten an. Erst im 5. Buch über das
  Staatseinkommen reproduziert er Ferguson. Ich habe in ``Misère de a
  Philosophie'' das Nötige über das historische Verhältnis von Ferguson,
  A. Smith, Lemontey und Say in ihrer Kritik der Teilung der Arbeit
  gegeben und dort auch zuerst die manufakturmäßig Teilung der Arbeit
  als spezifische Form der kapitalistischen Produktionsweise
  dargestellt. (l.c.p. 122 sq. \textless{}Siehe Band, S.
  145-147\textgreater{}) \textless{}=
\item
  Ferguson sagt bereits l.c.p. 281: ``Und das Denken selbst kann in
  diesem Zeitalter der Arbeitsteilungen zu einem besonderen Gewerbe
  werden.'' \textless{}=
\item
  G. Garnier, t. V seiner Übersetzung, p.~4-5. \textless{}=
\item
  Ramazzini, Professor der praktischen Medizin zu Padua, veröffentlichte
  1713 sein Werk ``De morbis artificum'', 1777 ins Französische
  übersetzt, wieder abgedruckt 1841 in der ``Encyclopédie des Sciences
  Médicales. 7me Div. Auteurs Classiques''. Die Periode der großen
  Industrie hat seinen Katalog der Arbeiterkrankheiten natürlich sehr
  vermehrt. Siehe u.a. ``Hygiène physique et morale de l'ouvrier dans
  les grandes villes en général, et dans la ville de Lyon en
  particulier''. Par le Dr.~A. L. Fonteret, Paris 1858, und {[}R. H.
  Rohatzsch,{]} ``Die Krankheiten, welche verschiednen Ständen, Altern
  und Geschlechtern eigenthümlich sind'', 6 Bände, Ulm 1840. Im Jahre
  1854 ernannte die Society of Arts eine Untersuchungskommission über
  industrielle Pathologie. Die Liste der von dieser Kommission
  gesammelten Dokumente findet man im Katalog des ``Twickenham Economic
  Museum''. Sehr wichtig die offiziellen ``Reports on Public Health''.
  Sieh auch Eduard Reich, M.D., ``Ueber die Entartung des Menschen'',
  Erlangen 1868. \textless{}=
\item
  ``To subdivide a man is to execute him, if he deserves the sentence,
  to assassinate him, if he does not \ldots{} the subdivision of labour
  is the assassination of a people.'' (D. Urquhart, ``Familiar Words'',
  London 1855, p.~119.) Hegel hatte sehr ketzerische Ansichten über die
  Teilung der Arbeit. ``Unter gebildeten Menschen kann man zunächst
  solche verstehn, die alles machen können, was andre tun'', sagt er in
  seiner Rechtsphilosophie. \textless{}=
\item
  Der gemütliche Glaube an das Erfindungsgenie, das der einzelne
  Kapitalist in der Teilung der Arbeit a priori ausübe, findet sich nur
  noch bei deutschen Professoren, wie Herrn Roscher z.B., der dem
  Kapitalisten, aus dessen Jupiterhaupt die Teilung der Arbeit fertig
  hervorspringe, zum Dank ``diverse Arbeitslöhne'' widmet. Die größre
  oder geringre Anwendung der Teilung der Arbeit hängt von der Länge der
  Börse ab, nicht von der Größe des Genies. \textless{}=
\item
  Mehr als A. Smith fixieren ältere Schriftsteller, wie Petty, wie der
  anonyme Verfasser der ``Advantages of the East-India Trade'' etc., den
  kapitalistischen Charakter der manufakturmäßigen Teilung der Arbeit.
  \textless{}=
\item
  Ausnahme unter den Modernen bilden einige Schriftsteller des 18.
  Jahrhunderts, die in bezug auf Teilung der Arbeit fast nur den Alten
  nachsprechen, wie Beccaria und James Harris. So Beccaria: ``Jedem
  beweist seine eigne Erfahrung, daß, wenn man Hand und Geist immer
  derselben Art von Arbeiten und Produkten zuwendet, man diese leichter,
  reichlicher und besser herstellt, als wenn jeder einzeln für sich das,
  was er benötigt, herstellen würde \ldots{} Auf diese Weise teilen sich
  die Menschen zum Nutzen der Allgemeinheit und zu ihrem eignen Vorteil
  in verschiedne Klassen und Stände.'' (Cesare Beccaria, ``Elementi di
  Econ. Publica'', ed. Custodi, Part. Moderna, t. XI, p.~28.) James
  Harris, später Earl of Malmesbury, berühmt durch die ``Diaries'' über
  seine Gesandtschaft in Petersburg, sagt selbst in einer Note zu seinem
  ``Dialogue concerning Happiness'', London 1741, später wieder
  abgedruckt in ``Three Treatises etc.'', 3. ed., Lond. 1772: ``Der
  ganze Beweis dafür, daß die Gesellschaft etwas Natürliches ist''
  (nämlich durch die ``Teilung der Beschäftigungen''), ``ist dem zweiten
  Buch von Platos''Republik" entnommen." \textless{}=
\item
  So in der Odyssee, XIV, 228: ``Denn ein andrer Mann ergötzt sich auch
  an andren Arbeiten'' und Archilochus beim Sextus Empiricus: ``Jeder
  erquickt seinen Sinn bei andrer Arbeit.'' \textless{}=
\item
  ``Poll' hpistato erga, kakwz d' hpistato panta''{[}griechsch: ``Poll'
  epistaio erga, kakos d'epistano panta.'' \textless{}``Viele Arbeiten
  konnt' er, doch alle konnt' er schlecht.''\textgreater{} - Der
  Athenienser fühlte sich als Warenproduzent dem Spartaner überlegen,
  weil dieser im Krieg wohl über Menschen, nicht aber über Geld verfügen
  könne, wie Thukydides den Perikles sagen läßt in der Rede, worin er
  die Athenienser zum Peloponnesischen Krieg aufstachet: ``Mit ihren
  Körpern Krieg zu führen sind die Selbstwirtschaftenden eher bereit als
  mit Geld.'' (Thuk., l. I, c. 141.) Dennoch blieb ihr Ideal, auch in
  der materiellen Produktion, die autarkeia {[}griechisch: autarkeia{]}
  , die der Teilung der Arbeit gegenübersteht, ``denn bei diesen gibt es
  Wohlstand, bei jenen aber auch die Unabhängigkeit''. Man muß dabei
  erwägen, daß es noch zur Zeit des Sturzes der 30 Tyrannen keine 5.000
  Athener ohne Grundeigentum gab. \textless{}=
\item
  Plato entwickelt die Teilung der Arbeit innerhalb des Gemeinwesens aus
  der Vielseitigkeit der Bedürfnisse und der Einseitigkeit der Anlagen
  der Individuen. Hauptgesichtspunkt bei ihm, daß der Arbeiter sich nach
  dem Werk richten müsse, nicht das Werk nach dem Arbeiter, was
  unvermeidlich, wenn er verschiedne Künste zugleich, also eine oder die
  andre als Nebenwerk treibe. ``Denn die Arbeit will nicht warten auf
  die freie Zeit dessen, der sie macht, sondern der Arbeiter muß sich an
  die Arbeit halten, aber nicht in leichtfertiger Weise. - Dies ist
  notwendig. - Daraus folgt also, daß man mehr von allem verfertigt und
  sowohl schöner als auch leichter, wenn einer nur eine Sache macht,
  seiner natürlichen Begabung gemäß und zur richtigen Zeit, frei von
  andern Geschäften.'' (``De Republica'', II, 2. ec., Baiter, Orelli
  etc.) Ähnlich bei Thukydides, l.c.c. 142: ``Das Seewesen ist eine
  Kunst so sehr wie irgend etwas andres und kann nicht bei etwa
  vorkommenden Fällen als Nebenwerk betrieben werden, sondern vielmehr
  nichts andres neben ihm als Nebenwerk.'' Muß das Werk, sagt Plato, auf
  den Arbeiter warten, so wird oft der kritische Zeitpunkt der
  Produktion verpaßt und das Machwerk verdorben, ``ergo u kairon
  diollutai'' {[}griechisch: ``ergou kairon diollytai.''
  \textless{}``die rechte Zeit für die Arbeit geht
  verloren''\textgreater{}{]} Dieselbe platonische Idee findet man
  wieder im Protest der englischen Bleichereibesitzer gegen die Klausel
  des Fabrikakts, die eine bestimmte Eßstunde für alle Arbeiter
  festsetzt. Ihr Geschäft könne sich nicht nach den Arbeitern richten,
  denn ``von den verschiedenen Operationen des Absengens, Waschens,
  Bleichens, Mangelns, Pressens und Färbens kann keine in einem
  bestimmten Augenblick ohne Gefahr der Schädigung abgebrochen werden
  \ldots{} Das Erzwingen derselben Essensstunde für alle Arbeiter kann
  gelegentlich wertvolle Güter dadurch in Gefahr bringen, daß der
  Arbeitsprozeß nicht beendet wird.'' Le platonisme où va-t-il se
  nicher! \textless{}Wo wird der Platonismus sich noch überall
  einnisten!\textgreater{} \textless{}=
\item
  Xenophon erzählt, es sei nicht nur ehrenvoll, Speisen von der Tafel
  des Perserkönigs zu erhalten, sondern diese Speisen seien auch viel
  schmackhafter als andre. ``Und dies ist nichts Wunderbares, denn wie
  die übrigen Künste in den großen Städten besonders vervollkommnet
  sind, ebenso werden die königlichen Speisen ganz eigens zubereitet.
  Denn in den kleinen Städten macht derselbe Bettstelle, Türe, Pflug,
  Tisch; oft baut er obendrein noch Häuser und ist zufrieden, wenn er
  selbst so eine für seinen Unterhalt ausreichende Kundschaft findet. Es
  ist rein unmöglich, daß ein Mensch, der so vielerlei treibt, alles gut
  mache. In den großen Städten aber, wo jeder einzelne viele Käufer
  findet, genügt auch ein Handwerk, um seinen Mann zu nähren. Ja oft
  gehört dazu nicht einmal ein ganzes Handwerk, sondern der eine macht
  Mannsschuhe, der andre Weiberschuhe. Hier und da lebt einer bloß vom
  Nähen, der andre vom Zuschneiden der Schuhe; der eine schneidet bloß
  Kleider zu, der andre setzt die Stücke nur zusammen. Notwendig ist es
  nun, daß der Verrichter der einfachsten Arbeit sie unbedingt auch am
  besten macht. Ebenso steht's mit der Kochkunst.'' (Xen., ``Cyrop.'',
  l. VIII, c. 2.) Die zu erzielende Güte des Gebrauchswerts wird hier
  ausschließlich fixiert, obgleich schon Xenophon die Stufenleiter der
  Arbeitsteilung vom Umfang des Markts abhängig weiß. \textless{}=
\item
  ``Er'' (Busiris) ``teilte alle in besondere Kasten \ldots{} befahl,
  daß immer die nämlichen die gleichen Geschäfte treiben sollten, weil
  er wußte, daß die, welche mit ihren Beschäftigungen wechseln, in
  keinem Geschäft gründlich werden; die aber, welche beständig bei
  denselben Beschäftigungen bleiben, jedes aufs vollendetste zustande
  bringen. Wirklich werden wir auch finden, daß sie in Beziehung auf
  Künste und Gewerbe ihre Rivalen mehr übertroffen haben als sonst der
  Meister den Stümper und in Beziehung auf die Einrichtung, wodurch sie
  die Königsherrschaft und übrige Staatsverfassung erhalten, so
  vortrefflich sind, daß die berühmten Philosophen, welche darüber zu
  sprechen unternehmen, die Staatsverfassung Ägyptens vor andren
  lobten.'' (Isokr., ``Busiris'',c. 8.) \textless{}=
\item
  cf.~Diod. Sic. \textless{}=
\item
  Ure, l.c.p. 20. \textless{}=
\item
  Das im Text Gesagte gilt viel mehr für England als für Frankreich und
  mehr für Frankreich als Holland. \textless{}=
\end{enumerate}
